
	首先誠摯的感謝指導教授陳明明博士,老師悉心的教導使我得以一窺WSN的深奧,不時的討論並指點我正確的方向,使我在這些年中獲益匪淺。老師對學問的嚴謹更是我輩學習的典範。
    本論文的完成另外亦得感謝老師們大力協助。因為有你們的體諒及幫忙,使得本論文能夠更完整而嚴謹。
 兩年裡的日子,實驗室裡共同的生活點滴,學術上的討論、言不及義的閒扯、讓人又愛又怕的宵夜、趕作業的革命情感、因為睡太晚而遮遮掩掩閃進實驗室........,感謝眾位學長姐、同學、學弟妹的共同砥礪,你/妳們的陪伴讓兩年的研究生活變得絢麗多彩。
	最後絕對不能忘記最了解、最支持我的家人─我的父親、母親及姊姊,在我喪失動力之時,隨時都能給予我心靈上無窮盡的關心與鼓勵,讓我有勇氣堅持到最後,完成研究的旅途。還有很多曾經幫助過我的朋友,因為有大家的幫助,我才能有今天的成果。想要感謝的人真的太多太多,就只有感謝上天了!




