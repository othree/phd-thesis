
台灣國家實驗研究院地震工程研究中心(國震中心)在九二一大地震後,與教育部合作執行「加速國中小老舊校舍及相關設備補強整建計畫」等計畫,評估全台灣的各級學校校舍之耐震能力,由於在這些計畫執行的過程中產生了大量的評估與調查資料,因此國震中心便建立了一個校舍耐震能力資料庫來收集各種相關的資料,收集了包括校舍的各種設計參數、材料強度、校舍現況及年齡、技師的評估與補強建議方案、實際補強的金額與補強方法等,資料涵蓋非常多元。此一資料庫所收集的校舍資料數量龐大,故除了當初設計的目的之外,應該還潛藏有其他類型的有用知識,但是難以由人工直接判斷取得,而資料探勘(Data Mining)就是用來分析這種數量龐大的資料,從中找出有用的潛藏知識的相關技術的統稱,本研究之目的即為利用資料探勘技術來發掘潛藏於此校舍耐震資料庫中的知識,從資料探勘的四種主要分析方法:回歸、分類、分群、關聯出發,分別探討各種方法在此資料庫中有何可能的分析方向,有哪些可能的潛藏知識,並進行分析,最後得到了三個有用且可靠度足夠的關係模型,分別為校舍資訊與耐震能力之關係模型、校舍資訊與破壞構件之關係模型以及校舍資訊與補強經費之關係模型。
	