%
% this file is encoded in utf-8
% v1.7
%  各符號以 \item[] 包住,然後接著寫說明
% 如果符號是數學符號,應以數學模式表示,以取得正確的字體
% 如果符號本身帶有方括號,則此符號可以用大括號 {} 包住保護
\begin{SymEntry}

\item[$A_C$]
校舍破壞地表加速度

\item[$A_D$]
校舍所需承受之~475~年週期最大地震之地表加速度

\item[$\alpha$]
命中率的容許誤差

\item[$CDR$]
耐震需求比

\item[$C$]
訓練模型時之懲罰值(Cost)參數

\item[$C_E$]
$Is$~值轉換之~$A_C$~等意義數值

\item[$ClaAc$]
一樓教室柱總斷面積

\item[$CorAc$]
一樓走廊外柱總斷面積

\item[$D\_isR$]
校舍是否需要補強

\item[$\epsilon$]
SVM~訓練模型之設定參數

\item[$InsAc$]
一樓隔間柱總斷面積

\item[$I$]
用途係數

\item[$I_s$]
耐震指標

\item[$Y_j$]
類神經網路中輸出層第~$j$~個處理單元之推論輸出值

\item[$Y$]
所建構模型中之輸出屬性

\item[$net_j$]
類神經網路中輸出層第~$j$~個處理單元之集成函數

\item[$W_{ij}$]
類神經網路中第~$i$~個輸入層單元與第~$j$~個輸出層單元間之連結權重

\item[$X_i$]
所建構模型中第~$i$~個輸入屬性,也可做為類神經網路中,輸入層第~$i$~個處理單元之輸入值

\item[$\theta_j$]
類神經網路中輸出層第~$j$~個處理單元之閥值

\item[$T_{adj}$]
$C_E$~值修正因子

\item[$T$]
建築物基本振動週期

\item[$T_0^D$]
$S_{D1}/S_{DS}$

\item[$NF$]
校舍樓層數

\item[$T_{AC}$]
柱等效強度

\item[$T_{AW}$]
牆等效強度

\item[$Af$]
總樓地板面積

\item[$S_{aD}$]
一般工址或近斷層區域之工址設計水平譜加速度係數

\item[$E$]
基本耐震性能

\item[$K$]
K-means~分群方法之初始群集數

\item[$k$]
$k$~群交叉驗證初始時,資料的分群數量

\item[$k'$]
Random subsampling~驗證方法中,隨機取樣並建立模型的次數

\item[$N$]
資料總數

\item[$NI$]
輸入參數的數量

\item[$NL$]
運算層的層數

\item[$N_g$]
需要最佳化的基因數量

\item[$O$]
偏移變數

\item[$P_1$]
樓層數,WGP~模型參數之一

\item[$P_2$]
鋼筋強度,WGP~模型參數之一

\item[$P_3$]
混凝土強度,WGP~模型參數之一

\item[$P_4$]
校舍長度,WGP~模型參數之一

\item[$P_5$]
校舍深度,WGP~模型參數之一

\item[$P_6$]
教室柱數量,WGP~模型參數之一

\item[$P_7$]
教室柱斷面積,WGP~模型參數之一

\item[$P_8$]
是否有走廊,WGP~模型參數之一

\item[$P_9$]
二樓樓地板面積,WGP~模型參數之一

\item[$P_{10}$]
總樓地板面積,WGP~模型參數之一

\item[$P_{11}$]
$S_{DS}$,校舍的震區設計水平譜加速度係數,WGP~模型參數之一

\item[$P_{12}$]
$S_{D1}$,一秒週期的設計譜加速度,WGP~模型參數之一

\item[$P_{13}$]
教室數量,WGP~模型參數之一

\item[$P_{14}$]
校舍建築物的跨距,WGP~模型參數之一

\item[$R^2$]
決定係數

\item[$S_i$]
第~$i$~個輸入參數的重要度

\item[$w_i$]
WGP~決策樹中,第~$i$~條分支線上之權重

\item[$\bar{C}$]
WGP~決策樹中之常數端點

\item[$x$]
輸入屬性

\item[$y$]
輸出屬性預測值

\item[$\hat{y}$]
輸出屬性實際值

\item[$q_1$]
平面及立面對稱性,初步評估調整因子之一

\item[$q_2$]
軟弱層顯著性,初步評估調整因子之一

\item[$q_3$]
裂縫鏽蝕滲水等程度,初步評估調整因子之一

\item[$q_4$]
變形程度,初步評估調整因子之一

\item[$q_5$]
平面耐震性,初步評估調整因子之一

\item[$q_6$]
短柱嚴重性,初步評估調整因子之一

\end{SymEntry}