
台灣國家實驗研究院地震工程研究中心(NCREE)在九二一大地震後,與教育部合作評估全台灣的各級學校校舍之耐震能力,在此計畫執行的過程中產生了大量的評估與調查資料,因此 NCREE 便建立了一個校舍耐震能力資料庫來收集各種相關的資料,收集了包括校舍的各種設計參數、材料強度、校舍現況及年齡、技師的評估與補強建議方案、實際補強的金額與補強方法等。收集的校舍資料數量龐大,除了當初設計的目的之外,應該還潛藏難以由人直接判斷取得的知識(knowledge)、模式(pattern)。資料探勘(Data Mining)就是用來分析這種數量龐大的資料,從中找出潛藏的知識的相關技術的統稱,本研究之目的即為利用資料探勘技術來發掘潛藏於此校舍耐震資料庫中的知識,本研究從資料探勘的四種主要分析方法:回歸、分類、分群、關聯出發,分別探討各種方法在此資料庫中有何可能的分析方向,有哪些可能的潛藏知識,並進行分析,最後得到了三個有用的預測模型,分別為校舍耐震能力預測模型、校舍破壞模式預測模型以及校舍補強經費預估模型。
	
In general, the aseismic ability of buildings is analyzed using nonlinear models. To obtain buildings’ aseismic abilities, numerical models are constructed based on the structural configuration and material properties of buildings, and their stress responses and behaviors are simulated. This method is complex, time-consuming, and should only be conducted by professionals. In the past, soft computing techniques have been applied in the construction field to predict the particular stress responses and behaviors; however, only a few studies have been made to predict specific properties of entire buildings. In this study, a Weighted Genetic Programming system is developed to construct the relation models between the aseismic capacity of school buildings, and their basic design parameters. This is based on information from the database of school buildings, as well as information regarding the aseismic capacity of school buildings analyzed using complete nonlinear methods. This system can be further applied to predict the aseismic capacity of the school buildings.