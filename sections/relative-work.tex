\renewcommand\thetable{\arabic{chapter}-\arabic{table}}
%\renewcommand\thefigure{\arabic{chapter}-\arabic{figure}}
\renewcommand{\theequation}{\arabic{chapter}-\arabic{equation}}
\chapter{相關研究}

本研究主要的目的是應用關聯式資料庫於輔助校舍耐震評估相關作業之資料管理與自動化,其包括了針對所有受評校舍其評估所需調查收集之結構耐震相關資料,以及其結構分析建模所需之各類結構構件非線性模型之構成資訊,以關聯式資料庫做有效的維護與管理,並為這些資料提供條件查詢及重複取出運用的使用模式,此外,亦建立了基於上述兩資料庫之校舍自動建模系統,以輔助校舍詳細評估之結構分析工作,故與本研究相關之研究回顧整理如下:

\begin{description}
  \item[應用關聯式資料庫於營建工程相關領域之研究]
  營建工程相關之研究,有想多的研究資料室需要先經由調查、實驗等工作才能收集到,而這些資料實驗或調查所得之資料,如果數量龐大時,都會使用資料庫系統整理並儲存,在營建領域上已有多位學者建置其各自專門領域相關研究之資料庫,例如國內有張鉅輝\cite{chang2003master}建置有混凝土橋梁耐震能力評估之資料庫系統;周武坤\cite{chou2002master}建置有高雄都會區地下管線工程管理資料庫;侯竣棕\cite{hou2000master}建置有南橫公路邊坡地工環境災害之資料庫等;蘇振綱\cite{su2000master}利用環境地質資料庫評估地理資訊系統計畫等。

  國外也有許多營建工程相關的資料庫,如~Law, K. H.~等人\cite{law1990management} 設計一通用於一般結構建築物之關聯式模型結構,此一模型結構可以表現一個結構建築物之各個構成元素以及其間之關係,不僅僅只有構成此一建築物基本的梁、柱、牆等,還包括內部隔間、窗戶、門等元素都可以使用此一模型結構儲存並表現相互的關係,而此模型結構也可以很容易的儲存進關聯式資料庫內,供後續之研究使用;Kim, C.-K.與 Lee, S.-E.\cite{kim2000study}使用關聯式模型設計了一個稱為~MLG model~的資料庫結構,將不同構件中相同的資訊獨立出來,除了可增加資料庫儲存結構物資訊的效率,還可以將幾何設計以外之資訊也整合進資料庫。

  而在營建領域相關之資料庫當中,數量最多的當屬紀錄材料、構件性質的資料庫了,如~Somo, S\cite{somo2006modeling}利用資料庫儲存大量~RC~梁之設計參數以及實驗結果,並利用這些設計參數以及不同的剪力容量計算方法計算出各梁可能的剪力容量,和實驗結果比較後找出較好之計算方法;Cuenca, E.~等人\cite{Cuenca20131089}~收集了~215~根~RC~樑的基本參數,包括了鋼筋比、預力、樑斷面積與長度的比等,並且還分析這些參數與其剪力行為間的關係;Hodkov{\'a}~等人\cite{hodkova2011envimat}~則建置了名為~Envimat.cz~的線上結構建材資料庫,可以讓使用者在線上查詢各種不同建築材料的性質和對環境的影響,使用者也可以新增或編輯資料,讓資料庫內的資料品質能夠逐步提升;Louren{\c{c}}o~等人\cite{lourencco2013simplified}~則建置了一個資料庫收集了西班牙、葡萄牙和義大利境內~44~棟磚造古蹟教堂,各六個簡易耐震能力評估參考索引,這些索引的可靠性均有驗證過,使用的驗證資料是紐西蘭~2012~年~2~月~22~日的地震資料;Dat Duthinh~和~William P. Fritz\cite{duthinh2006nonlinear}~先使用風洞試驗,將一個低層鋼結構的模型加上各種不同角度方向之風力,建立了一個資料庫收集試驗資料,並從中挑出鋼結構所受最大力矩的案例,和~ASCE 7-02\cite{asce2002minimum}~規範所建議之風力比較,結果可以設計出只需要多出~3.6\%~的鋼材,就可以增加對風力的抵抗能力達~30\%。

  \item[電腦輔助耐震評估之研究]
  趙宜峰\cite{chou2004master}提出一套網路式之橋梁耐震評估系統,所建立之橋梁耐震評估系統主要由初步評估網路式資料庫與橋梁耐震功能分析程式所組成,網路式之建置可讓工程師不需安裝任何軟體即可直接透過網際網路來與伺服端電腦連線使用本系統,方便地透過網頁瀏覽器上網進行橋梁評估。使用之評估流程是先由橋梁耐震初步評估網路式資料庫快速將橋梁依其耐震能力及重要性評分,以得分作為評估者進一步進行詳細耐震能力評估或進行補強之依據。橋梁初步評估後若發現有落橋之可能時,則可直接透過網路使用橋梁細部評估程式,以圖形化的介面,進行橋梁耐震功能分析,評估方式為將結構側推頻譜、剪力容量頻譜、地震需求量頻譜,依結構韌性進行迭代,得到彎矩功能點及剪力破壞點,以作為工程師判斷之依據。張瑜晏\cite{chang2004master}則是建置出一套建築物耐震能力詳細評估輔助系統,命名為~ACES,可自動完成定義塑鉸性質、指定塑鉸位置與定義側推分析例之步驟,然後經~ETABS 8~分析模組完成結構容量震譜後,再透過後處理模組之解析,自動完成結構性能點之求取,並同時得到此性能點所對應之地表加速度~$Ac$~值。鄧彬斌\cite{deng2005phd}]利用一建築物模型樣板,透過自我學習機制自動的產生大量不同的校舍建築模型參數,如樓層數、樓高、鋼筋強度等,並透過分析這些校舍建築模型之耐震詳細評估數據,找出對校舍耐震能力影響較大之參數。

  \item[資料探勘於營建工程領域之研究]
  張玉瑛\cite{chang2007master}用「支援向量機」(Support Vector Machine, SVM)做為預測工具,其不需先提供充足的資訊範圍及預設任何的假設,且正確性不易受到資料量大小之限制;並於預測模型中納入影響經濟環境的變數,使得預測模式在經濟環境改變時,仍保有其正確性。研究結果發現,於有限的資料樣本中加入經濟環境的變數,藉由「支援向量機」所建立之模型,可作出與實際指數趨勢相當的預測結果,證明所採用的方法之適用性與正確性。


  \item[應用關聯式資料庫於校舍之研究]


\end{description}