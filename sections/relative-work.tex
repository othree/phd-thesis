\renewcommand\thetable{\arabic{chapter}-\arabic{table}}
%\renewcommand\thefigure{\arabic{chapter}-\arabic{figure}}
\renewcommand{\theequation}{\arabic{chapter}-\arabic{equation}}
\chapter{相關研究}

本研究主要的目的是應用關聯式資料庫於輔助校舍耐震評估相關作業之資料管理與自動化,其包括了針對所有受評校舍其評估所需調查收集之結構耐震相關資料,以及其結構分析建模所需之各類結構構件非線性模型之構成資訊,以關聯式資料庫做有效的維護與管理,並為這些資料提供條件查詢及重複取出運用的使用模式,此外,亦建立了基於上述兩資料庫之校舍自動建模系統,以輔助校舍詳細評估之結構分析工作,故將與本研究相關之研究回顧整理如為三大類:應用關聯式資料庫於營建工程相關領域之研究、資料探勘於營建工程領域之研究以及校舍耐震資料庫與資料探勘之研究,以下各章節分別介紹之。

\section{應用關聯式資料庫於營建工程相關領域之研究}
 
營建工程相關之研究,有許多的研究資料室需要先經由調查、實驗等工作才能收集到,而這些資料實驗或調查所得之資料,如果數量龐大時,都會使用資料庫系統整理並儲存,在營建領域上已有多位學者建置其各自專門領域相關研究之資料庫,例如國內有張鉅輝\cite{chang2003master}建置有混凝土橋梁耐震能力評估之資料庫系統;周武坤\cite{chou2002master}建置有高雄都會區地下管線工程管理資料庫;侯竣棕\cite{hou2000master}建置有南橫公路邊坡地工環境災害之資料庫等;蘇振綱\cite{su2000master}利用環境地質資料庫評估地理資訊系統計畫等。

國外也有許多營建工程相關的資料庫,如~Law, K. H.~等人\cite{law1990management} 設計一通用於一般結構建築物之關聯式模型結構,此一模型結構可以表現一個結構建築物之各個構成元素以及其間之關係,不僅僅只有構成此一建築物基本的梁、柱、牆等,還包括內部隔間、窗戶、門等元素都可以使用此一模型結構儲存並表現相互的關係,而此模型結構也可以很容易的儲存進關聯式資料庫內,供後續之研究使用;Kim, C.-K.與 Lee, S.-E.\cite{kim2000study}使用關聯式模型設計了一個稱為~MLG model~的資料庫結構,將不同構件中相同的資訊獨立出來,除了可增加資料庫儲存結構物資訊的效率,還可以將幾何設計以外之資訊也整合進資料庫。

而在營建領域相關之資料庫當中,數量最多的當屬紀錄材料、構件性質的資料庫了,如~Somo, S\cite{somo2006modeling}利用資料庫儲存大量~RC~梁之設計參數以及實驗結果,並利用這些設計參數以及不同的剪力容量計算方法計算出各梁可能的剪力容量,和實驗結果比較後找出較好之計算方法;Cuenca, E.~等人\cite{Cuenca20131089}~收集了~215~根~RC~樑的基本參數,包括了鋼筋比、預力、樑斷面積與長度的比等,並且還分析這些參數與其剪力行為間的關係;Hodkov{\'a}~等人\cite{hodkova2011envimat}~則建置了名為~Envimat.cz~的線上結構建材資料庫,可以讓使用者在線上查詢各種不同建築材料的性質和對環境的影響,使用者也可以新增或編輯資料,讓資料庫內的資料品質能夠逐步提升;Louren{\c{c}}o~等人\cite{lourencco2013simplified}~則建置了一個資料庫收集了西班牙、葡萄牙和義大利境內~44~棟磚造古蹟教堂,各六個簡易耐震能力評估參考索引,這些索引的可靠性均有驗證過,使用的驗證資料是紐西蘭~2012~年~2~月~22~日的地震資料;Dat Duthinh~和~William P. Fritz\cite{duthinh2006nonlinear}~先使用風洞試驗,將一個低層鋼結構的模型加上各種不同角度方向之風力,建立了一個資料庫收集試驗資料,並從中挑出鋼結構所受最大力矩的案例,和~ASCE 7-02\cite{asce2002minimum}~規範所建議之風力比較,結果可以設計出只需要多出~3.6\%~的鋼材,就可以增加對風力的抵抗能力達~30\%。

  %\item[電腦輔助耐震評估之研究]
  %趙宜峰\cite{chou2004master}提出一套網路式之橋梁耐震評估系統,所建立之橋梁耐震評估系統主要由初步評估網路式資料庫與橋梁耐震功能分析程式所組成,網路式之建置可讓工程師不需安裝任何軟體即可直接透過網際網路來與伺服端電腦連線使用本系統,方便地透過網頁瀏覽器上網進行橋梁評估。使用之評估流程是先由橋梁耐震初步評估網路式資料庫快速將橋梁依其耐震能力及重要性評分,以得分作為評估者進一步進行詳細耐震能力評估或進行補強之依據。橋梁初步評估後若發現有落橋之可能時,則可直接透過網路使用橋梁細部評估程式,以圖形化的介面,進行橋梁耐震功能分析,評估方式為將結構側推頻譜、剪力容量頻譜、地震需求量頻譜,依結構韌性進行迭代,得到彎矩功能點及剪力破壞點,以作為工程師判斷之依據。張瑜晏\cite{chang2004master}則是建置出一套建築物耐震能力詳細評估輔助系統,命名為~ACES,可自動完成定義塑鉸性質、指定塑鉸位置與定義側推分析例之步驟,然後經~ETABS 8~分析模組完成結構容量震譜後,再透過後處理模組之解析,自動完成結構性能點之求取,並同時得到此性能點所對應之地表加速度~$Ac$~值。鄧彬斌\cite{deng2005phd}利用一建築物模型樣板,透過自我學習機制自動的產生大量不同的校舍建築模型參數,如樓層數、樓高、鋼筋強度等,並透過分析這些校舍建築模型之耐震詳細評估數據,找出對校舍耐震能力影響較大之參數。

\section{資料探勘於營建工程領域之研究}

資料探勘在營建工程領域也已經有相當多的應用,主要原因在於營建工程領域有很多非線性的問題,很適合使用資料探勘這種軟式運算的方法來解答,而其中最為被廣泛應用的當屬類神經網路了,Golafshani~等人\cite{golafshani2014artificial}~分析了上百根~GFRP~筋混凝土的介面連結強度,並用類神經網路和基因演算法建置出該強度與其設計參數間的關係模型,表現都非常好,也比~ACI\cite{aci2006guide}~規範的公式要來的準確;Arslan\cite{arslan2010evaluation}~在土耳其收集了~256~棟四層或是七層樓高,符合當地耐震能力規範~TEC-2007~的~RC~建築物,挑選了~8~個關鍵屬性包括樓層樹、混凝土強度、鋼筋強度、剪力牆比例、強柱弱樑形式、是否有短住等,並使用類神經網路建立了這些屬性與建築物受到側推力時的總位移量間之關係模型,並使用~SAP 2000~所做之非線性分析結果比較;Mohammad~等人\cite{azadi2013assessment}~則是探討地底隧道對與地面建築物造成的沈陷量,和使用有限元素分析方法得到的數值相比差距不大,並且提出了一個計算公式,根據公式內使用的參數也可以得知,建築物的沈陷量和隧道大小、與建築物的水平、垂直距離以及建築物的規模有相當明確的關係;Elshafey~等人\cite{elshafey2013predicting}~則使用類神經網路建立了一個混凝土表面裂縫間距的預測模型,並且也提出一個新的計算公式;Sarıdemir\cite{saridemir2009prediction}~則是利用類神經網路來預測添加了矽石和高嶺土的混凝土抗壓強度;Bekir~和~Mustafa\cite{topccu2008prediction}~則使用類神經網路和橡膠粉混凝土其製成比例的資訊來預測其單位中和流動性;Ince\cite{ince2004prediction}~則利用類神經網路來預測混凝土的斷裂參數(fracture parameter)。

除了類神經網路外,尚有許多其他的資料探勘演算法被應用在營建工程領域,例如張玉瑛\cite{chang2007master}用支持向量機(Support Vector Machine, SVM)做為預測工具,其不需先提供充足的資訊範圍及預設任何的假設,且正確性不易受到資料量大小之限制,並於預測模型中納入影響經濟環境的變數,使得預測模式在經濟環境改變時,仍保有其正確性。研究結果發現,於有限的資料樣本中加入經濟環境的變數,藉由~SVM~所建立之模型,可作出與實際指數趨勢相當的預測結果,證明所採用的方法之適用性與正確性;而由於國際間匯率經常變動,仰賴進口建材的國家之供應商很可能會因為匯率的變動而蒙受巨大損失,因此~Chen~與~Lin\cite{chen2010developing}~也使用~SVM~做為工具,建立一個預測模型,可以建議供應商是否需要使用衍生性金融商品避險;Daniel~和~Moncef\cite{tuhus2010genetic}~則將公寓大樓外殼的設計歸納為~9~個參數,並且使用基因演算法(Genetic Algorithm, GA)最佳化公寓大樓外殼的設計,藉以達到最少的能源消耗。Kaveh~等人\cite{kaveh2010performance}~則是使用蟻群演算法(Ant Colony Optimization, ACO)來最佳化鋼結構的設計,盡量減少鋼材的使用但是能夠符合規範,結果在同樣條件下,使用~ACO~方法得到的設計比~GA~最佳化設計所使用的鋼材還要來的少許多。

\section{校舍耐震資料庫與資料探勘之研究}

近幾年也越來越多關於校舍建築物的資料收集和研究,以本研究所使用的校舍耐震資料庫來說,除本研究外尚有陳鴻銘與高偉格\cite{chen2008computer}~利用耐震資料庫中的校舍幾何設計參數,搭配另外一個結構構件模型資料庫,透過網頁介面就可以讓使用者快速的產生可用於非線性分析的校舍數值模型;鄭明淵等人\cite{chen2012seismic}~則結合支持向量機(SVM)以及快速混雜基因演算法(fmGA)發展一演化式支持向量機推論系統(ESIS),並使用這個系統建立典型校舍的性能目標地表加速度與校舍幾何參數間的關係模型,這些參數包括樓層樹、總樓地板面積、各種牆斷面積、柱斷面積等。

除國內外,國外也有相關的研究,de Santoli~等人\cite{de2014energy}~建立了羅馬的學校校舍資料庫,而其主要目標在於分析現在的校舍能源效率,並研究如何補強耗能高的建築成為綠建築,減少能源的浪費;日本和台灣同處環太平洋地震帶,對於學校校舍受地震的災損分析也有相關研究,例如:SATO~和~MINAMI\cite{SATOShinji:2004-12}~就針對~2001~年芸予地震對於日本廣島縣校舍受所造成的損害進行了詳細的資料收集和分析。

Jafarzadeh~等人\cite{jafarzadeh2014seismic}~針對建築物的補強經費預估為目標,收集了~158~棟學校校舍的補強計畫資料,收集的資料則是以能構件力補強經費預估模型為目標,收集每棟建築物影響耐震補強經費的~14~種屬性如下:

  \begin{multicols}{2}
  \begin{itemize}
  \item 總樓地板面積
  \item 樓層數
  \item 屋齡
  \item 載重(靜載重 + 活載重)
  \item 平面對稱性
  \item 垂直對稱性
  \item 樓版形式
  \item 基礎形式
  \item 結構形式
  \item 非結構構件與設備
  \item 地震潛勢
  \item 土壤形式
  \item 減載措施
  \item 耐震需求
  \end{itemize}
  \end{multicols}

這些屬性都是對於補強經費影響很高的屬性\cite{jafarzadeh2012seismic},是~Jafarzadeh~等人花了九個月的時間在伊朗逐步調查取得的,資料的可靠度非常高,而補強經費的部分則還細分為結構補強經費(Seismic Retrofit Construction Cost, SRCC)、非結構補強經費(Non-construction Cost)和非直接經費(Indirect Cost);~Jafarzadeh~等人後續接著使用這些資料,建立了校舍的耐震能力補強經費的預測模型,分別使用了多元迴歸分析\cite{jafarzadeh2013predicting}和類神經網路\cite{jafarzadeh2013application},預測的目標則為補強結構經費淨值(Retrofic Net Construction Cost, RNCC)其為~Iran Construction Cost Index(ICCI)所定義,RNCC~乘上九個調整因子後即為結構補強經費,其公式如下:

    \begin{equation}SRCC = RNCC \times \prod_{i=1}^9 (\text{Coefficient})_i \end{equation} 

這九個調整因子包括了:間接成本、地理位置、樓層數、建築物高、機電設備更新、意外開銷、物價波動、計畫複雜度和基礎完整度,所以~RNCC~可以視為完全去除各種社會因素和不可抗力造成的經費變化,與其相關的參數就只剩下建築物結構、狀況等和結構強度直接相關的參數,非常適合做為耐震能力補強經費預測模型的預測目標,Jafarzadeh~等人所建立的校舍補強經費預估模型,表現最好的是使用類神經網路所建立的模型,$R^2$~可以達到~0.842,使用了~5~個參數,分別為總樓地板面積、樓層數、地震潛勢、土壤形式和結構形式,但是其測試集的表現較不好,$R^2$~只有~0.433;使用多元線性迴歸所建立的預測模型雖然表現較差,表現最好的一組屬性集之~$R^2$~為~0.710~,使用到的屬性則為總樓地板面積、樓層數、單位面積載重、地震潛勢、土壤形式、平面對稱性、結構形式和是否有四面圍束磚牆等七個屬性,其測試集之~$R^2$~表現則與訓練集不相上下,可靠度較使用類神經網路所建立的預測模型要高。


