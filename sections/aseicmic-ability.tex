\renewcommand\thetable{\arabic{chapter}-\arabic{table}}
%\renewcommand\thefigure{\arabic{chapter}-\arabic{figure}} 
\chapter{校舍資訊與耐震能力之關係模型}

在校舍耐震資料庫中,不同階段的調查資料有不同的校舍耐震能力指標,這些指標都是以數值形式來量化校舍建築物的耐震能力,有了此一數值後,可以快速找出有安全疑慮的校舍,根據耐震能力排序,甚至推算會需要多少預算來補強多少棟校舍等等,都可以使用此數值作為依據來達成,可以說是非常重要的校舍特性,然而要取得此一數值非常耗時耗力,如果有其他快速便宜的方法可以取得此一數值,那便可以大幅度的減少校舍耐震能力補強作業的所需要的時間與經費,因此本研究的第一個資料探勘目標,便是找出預測耐震能力索引的預測模型,得到此一預測模型後,便可以根據校舍的設計參數與現況作為輸入參數,快速的得到該校舍的耐震能力索引參考值。

初步評估階段資料的耐震能力指標是 $Is$ 值,此一數值為專業人士於現場良策、調查後,使用國家地震中心的專家根據過往的實驗數據統計分析後,所設計出的一個評估表計算而得到的,為一初步的評估值,較詳細評估所計算的耐震能力指標可靠度來的低,但仍是校舍補強計畫初期的篩選工作中非常重要的數值。詳細評估和補強設計階段的耐震能力指標則是 $CDR$ 值,$CDR$ 值是由專業人士根據校舍的設計與實際狀況建製結構模型,並進行非線性的推垮分析所得到的,此數值與建築物之設計參數為高度非線性的關係,也是最接近校舍建築物實際耐震能力的量化指標。

而除了數值量化的耐震能力指標外,本研究還將「校舍是否需要補強」($D\_isR$)這個二元指標作為耐震能力來作為校舍耐震資料庫中的第三個耐震能力指標,$Is$、$CDR$、$D\_isR$ 這三個耐震能力指標的預測模型,即為本研究所取得的第一個校舍耐震資料庫隱含知識,以下便針對不同指標的預測模型的資料探勘流程與結果詳細介紹。

\section{Is 值與校舍設計之關係模型}

校舍耐震資料庫中的初步評估資料表的耐震能力索引是為 $Is$ 值,初步評估是校舍耐震能力補強作業中,很重要的篩選過程,因為建築物的詳細評估所需要的金額極高,評估需要的時間也很長久,因此需要一個速度、價錢和可靠度都可以接受的評估方法,用比較短的時間找出耐震能力有疑慮的校舍,盡早處理。而地震中心所設計的初步評估表即為一個速度、價錢和可靠度都可以接受的評估方法,只要根據建築物的設計參數,像是樓層數、長度、深度、柱子尺寸以及校舍現況等,就可以快速的得到一個極具參考價值的耐震能力指標,而其計算的原理也與詳細評估的耐震能力指標索引 $CDR$ 值一樣,為耐震容量與耐震需求的比。找出 $Is$ 值與校舍設計參數之間的關係模型,可以進一步分析初步評估表中各種數據的重要性,並將結果回饋到此一初步評估表,將初步評估表調整的更為精簡,負責評估的人員也可以更快速的完成此意評估表格。

The proposed prediction model, based on data-mining technology, has six steps – business understanding, data understanding, data preparation, modeling, evaluation, and deployment – as recommended in the cross industry standard process for data mining (CRISP-DM) proposed by Chapman et al. (2000) (Fig. 9). The objective of this work is to construct a prediction model for the aseismic ability indices of school buildings and collect complete data. Hence, the major tasks are data preparation, modeling, and evaluation. Fig. 10 shows the procedural plan based on needs and the methods chosen. 

\subsection{資料前處理}

首先,從國震中心的校舍耐震資料庫中,將各棟校舍的 $Is$ 值與其校舍的相關資料挑出並整合在一起,第一步是過濾掉明顯不合理的錯誤資料,這些資料通常是人為的錯誤造成的,而過濾的方法則是使用國陣中心建議的過濾條件:

\begin{itemize}
\item 校舍總長或總深度超過兩百公尺
\item 有任意牆厚度超過五十公分
\item 有任意柱的長或寬超過一百公分
\item 柱間垮距超過八公尺或小於兩公尺
\end{itemize}

大約有八百筆資料符合這組基本的過濾條件,而在初步的過濾之後,前處理的第二個步驟則是減少資料屬性的維度,主成分分析(Principal component analysis)是很常使用的資料前處理方法,這個分析方法可以找出所有資料屬性中,重要度較高的屬性,它是把原始的多個資料屬性,透過向量轉換的方式,線性組合出主要的資料屬性。

Roughly 800 bits of valid data are obtained. After filtering using these conditions. Principal component analysis (PCA) is then used to reduce the dimensionality of data attributes. Notably, PCA, a very common data preparation method, can identify very important attributes among various attributes. The goal is to convert the original variables through vector transition into mutually independent variables of a linear combination. The ideal situation is that principal components obtained from linear combination retain most of the information of original variables.

校舍的分群則是基於校舍的設計模式,目標是要找出校舍建築物的幾種特定設計模式,也就是要分析校舍的幾何設計參數、柱量、牆量等等,經過不斷的測試及調整,最後找出了五個主要成分屬性:

\begin{itemize}
\item 走廊柱資訊
\item 教室柱資訊
\item 強設計資訊
\item 牆量資訊
\item 柱量資訊
\end{itemize}


Clustering analysis of school buildings is based on the design patterns of school buildings; the goal is to find several design patterns. Hence, the attributes for analysis are the geometric information of school buildings, such as dimensions and quantity of walls and columns, and width and height of school buildings. Other attributes, such as year of construction and locale, may not be incorporated into PCA analysis. After continuous testing and adjustment, five major attributes are obtained and the importance of constitutional fields is considered as the basis for naming. The five attributes are as follows: corridor column information; classroom column design information; wall design information; data for number of walls; and, data for number of classroom columns.



在前面的處理完成後,還使用分群分析將校舍分群,以幫助後續的 $Is$ 值關係模型的建立,分群的目標是將校舍依照不同的設計模式特性區分開來,將相近設計的校舍放在同一群集,分群的依據是主成分分析法所找出的五種主要校舍設計相關屬性,圖十一是使用 SPSS Clemitine 進行此一分群分析時的節點設計圖,分別使用的 K-means 和 Two-Step 兩種分群演算法,其參數設定如下:

After data preparation, clustering analysis is utilized to find hidden design patterns of school buildings. This study uses the K-means and two-step clustering methods. Fig. 11 shows the node deployment for clustering using the SPSS Clementine (2007) software. The parameter choices for the two methods are as follows.

\subsubsection{K-means}

此一分群方法最重要的設定參數是初始的群集數 $k$,雖然有許多研究都在研究如何找出最佳的 $k$ 值,但是目前仍沒有一個方法可以宣稱它找到 $k$ 值是最佳的,唯有對該領域的專業了解以及詳細的測試才能得到最佳的 $k$ 值,在本分析案例中,K-means 分群的停止條件是設定為 20 次迭代,如果 $k$ 太大,那會造成分群結果無法在 20 次迭代內收斂,如果 $k$ 小於 6,則分群的結果就可以在 20 次迭代內完整的收斂,每筆資料都會故底定在所屬的群集內,不在變動,最後挑選的 $k$ 值是 5,而根據此意參數的分群結果可以找出三個主要的校舍群集,分別的比例是 28\%、56\% 和 16\%。

The most important parameter with this clustering method is the initial group, k. Although many researchers have developed methods for choosing the initial K value, no method can confirm that it has found the best K value. The best K can be found only based on researcher understanding and problem testing. In this study, K-means clustering is set to stop after 20 iterations. If the K value is too large and cannot be completely converged after 20 operations, some data points will continue to change the clusters to which they belong. When the K value is reduced to <6, Clustering models can be completely converged after 20 iterations, become stable, and no longer change cluster label of all data bit. The K value chosen is 5, and three major clusters are obtained. The remaining two clusters have few data and are considered outliers. The distribution of three major clusters are 28\%, 56\%, and 16\%.

\subsubsection{Two-Step}

Two-Step 分群有兩個優點,一是複雜度不高,運算時間與資料數量間之關係為線性關係,第二個優點就是不需要由人工決定分群的群數,演算法即可自己根據資料狀況決定,操作人員只需給予上下限,在本分析中,上下限的設定為最少兩個群集,最多八個群集,而最後的分析結果是所有的資料都被分到兩個群集中,分別佔了 54\% 和 45\%。

This clustering method has two features. The first is enhanced scalability. The algorithm has low complexity. Computing time does not grow nonlinearly as data volume increases. The other feature is that it can determine the number of clusters, unlike K-means clustering, which requires manual designation of parameters. However, this work can designate the upper and lower limits for the number of clusters. This work sets the limit to 2–8 clusters based on experience with K-means clustering. Consequently, all data are divided into two clusters, accounting for 54\% and 45\% of all data.






此資料探勘分析還用了十群交叉驗證來驗證結果的可靠度,因此資料前處理的最後一個步驟就是將整理好的資料隨機分為十組。

This work uses 10-fold cross validation to validate the prediction model. After preparation, data are grouped by first dividing data randomly and equally into 10 clusters. One cluster is then chosen as a dataset for validation and the remaining nine clusters are combined into one training dataset.

\subsection{資料探勘}

完成資料前處理,將校舍的群集分好之後,才開始建立 $Is$ 值與校舍設計參數的關係模型,本分析使用了廣義線性模型、線性回歸和類神經網路三種分析方法,每種方法都有三種分析資料群,分別為先經過 K-means 分群的資料、先經過 Two-step 分群的資料及沒有先經過分群的資料。圖 12 為使用 SPSS Clemitine 分析時的節點設計圖。以下分別對三種分析方法的參數設定作說明。

The prediction model is built only after clustering is completed. This work uses a GLM, simple regression, and ANNs to build the prediction model based on three groups of data – not clustered in advance, clustered by K-means, and clustered by the two-step method in advance. In total, nine prediction models are generated. Fig. 12 shows the node configuration within SPSS Clementine. Below are the parameters chosen for the three methods. 

\subsubsection{Generalized linear model}

廣義線性模型是假設在輸入參數和預測目邊之間有一個可以用連結函數表達的關係,這個連結函數可能是指數函數、對數函數、Logistic 函數等,經由一些測試資料的測試,我們選擇使用對數函數作為連結函數,並且根據實際的資料分布選擇了常態分布作為輸入參數的分布函數,而關係模型的分布也選擇常態分布,因其表現教其他分布形式較好。

The GLM assumes a relationship between input variables and a predictor; this relationship can be built by a link function such as identity function, log function, logit function, or power function. After available link functions testing on some data, the prediction model performs best when using log function as the link function; hence, this work chose the log function. The distribution function of the predictor is based on the actual distribution of data. This work chose the normal distribution, which is close to the real data distribution. The prediction model constructed using the normal distribution performed better than those with other distribution types.

\subsubsection{Simple regression}

線性回歸是選擇使用最小平方根法來建立校舍建築物的設計參數與其耐震能力 $Is$ 之間的關係,這也是最常使用的回歸方法之一。

Simple regression in this work uses the least square method by adopting the building design parameters as independent variables $X$ and the aseismic ability of buildings as dependant variables $Y$. The linear equation between regressed design parameters and aseismic indices serve as the model for predicting aseismic ability $Y$ based on building design parameters $X$.

\subsubsection{Artificial neural networks}

類神經網路需要決定的參數包括隱藏層的數量、每層的神經元數量、學習率、停止條件等,除了直接設定神經網路的參數,還有一些方法可以使用,例如 dynamic、multiple 或是 prune method 可以用來調整並找出最佳的神經網路大小和結構,dynamic method 是從一個小型的神經網路開始(兩個隱藏層、每層兩個神經元),慢慢成長,並且比較成長前後的神經網路效能與結果,Multiple method 則是同時產生各種不同的神經網路,並且一起訓練到達停止條件,然後在從中挑選出表現最好的一個,而 Prune Method 則是從一個大的類神經網路開始,慢慢的把重要度低的神經元節點拿掉。本分析最後挑選的是 Exhaustive Prune Method,是 Prune Method 的一種修改形式,對於節點的篩選要求較高,是所有方法中最花時間的,但是通常也可以找到最好的結果。其他的類神經網路設定參數為:初始的神經網路為兩層隱藏層,其中一層有 30 個神經元、一層有 20 個神經元,停止條件為 250 個訓練循環,在這個設定下,Exhaustive Prune Method 是表現最好的方法。

Generally, ANNs must decide on such parameters as number of hidden layers, number of neurons in each layer, learning rate, and stop condition. Aside from directly setting these parameters, methods such as dynamic, multiple, and Prune methods are available for adjusting and finding the optimal size and structure of the neural network. The dynamic method starts with a small neural network (two hidden layers with two neurons for each layer), expands network size gradually, and decides on the further expansion based on model performance before and after expansion. The multiple methods constructs multiple neural networks simultaneously, trains all neural networks to reach the ``stop condition,'' and then selects the group with the best performance. In contrast with the dynamic method, which slowly builds a large neural network from a small one, the Prune method first builds a large network and then removes neurons with low importance based on training. This work chooses the exhaustive Prune method, a special application of the Prune method. The initial neural network has two hidden layers, one with 30 neurons and the other with 20 neurons. The stop condition is set to 250 training cycles. Under this limit, the prediction model built by the exhaustive Prune method performs best.


\subsection{驗證}

驗證有兩個主要的目的,一是確保資料探勘找到的關係模型的可靠度,而不會找到只適用於該組訓練資料集的關係模型,第二個目的是可以用來作為比較不同分析方法的指標數據,本分析使用的驗證方式是十群交叉驗證,這個方法將所有的資料等分成十份,每次挑選九組出來作為訓練資料集,留下一組作為驗證資料集,如此可以得到十組模型以及其可靠度的指標,求此十組指標平均值即可得到代表此關係模型的可靠度代表值,而本分析所選擇的指標有三個,線性關係、絕對平均誤差(Mean Absolute Prediction Error, MAPE)以及 hit rate。

Validation work has two purposes. The first is to ensure model reliability instead of to generate only good performance during data training. The second is to serve as a benchmark for comparing the performance of different prediction models. This work uses 10-fold cross validation to assess and compare the performance of prediction models. This method divides a fixed amount of data into 10 groups, conducts 10 rounds of model building and validation, chooses a different group of data for testing, trains the model with remaining nine groups of data, and uses test group data to validate model accuracy. After validation for 10 times, the accuracy of the 10 models is obtained and their average is taken as the accuracy of this algorithm. This study uses linear correlation, mean absolute prediction error (MAPE), and hit rate as the indices for comparing the prediction model performance.

\subsection{結果}

此資料探勘分析會產生九個校舍耐震能力與校舍建築設計資訊間的關係模型,包括直接使用GLM、線性回歸和類神經網路的三組,混合使用 K-means 或是 Two-Step 分群的六組,要比較其優劣我們使用了線性關係 $R$ 、 MAPE 和 hit rate 三個指標,並配合十群交叉驗證,其中線性關係 $R$ 之公式為:

This work constructed nine prediction models; three are directly generated by the GLM, simple regression, and ANNs. The mixed model of K-means and two-step clustering generated three prediction models. Hence, nine models were obtained and 10-fold cross validation is used to compare the performance of the three reference indices – R2, MAPE, and hit rate. Notably, R2, the linear correlation is

\begin{equation} R = \dfrac{\sum{(\hat{y_i} - \tilde{y})^2}}{\sum{(y_i - \tilde{y})^2}} \label{eq:RSQ}\end{equation} 

其中 $y_i$ 是校舍的實際耐震能力指標 $Is$ 值, $\hat{y_i}$ 則是透過此關係模型的到的推估耐震能力指標值,$\tilde{y}$ 則是所有資料的 $Is$ 值之平均,透過此一公式即可得到實際的 $Is$ 值與透過關係模型得到的推估值之間的線性關係,線性關係越高表示兩者之間越接近,也代表著關係模型的正確性。MAPE 的公式為:

where yi is the aseismic CDR of school buildings obtained using nonlinear analysis of the database, yi is the CDR obtained from the prediction model, and y~ is the average aseismic CDR of school 651 buildings obtained using nonlinear analysis. The correlation between aseismic CDR of school buildings obtained via the prediction model and nonlinear analysis can be determined based on linear correlation. A high CDR indicates a strong correlation and many opportunities to make correct predictions. The MAPE is derived as

\begin{equation} MAPE = \dfrac{\sum{\dfrac{y_i - \hat{y_i}}{y_i}}}{N} \label{eq:MAPE}\end{equation} 

其中 $N$ 是資料總數,MAPE 的是用來表示關係模型誤差之數值,由於關係模型不可能完全沒有誤差,即使有非常高的線性關係也是會有誤差,因此會使用 MAPE 作為判斷其誤差程度的參考。 hit rate 的定義是:

where N is the number of samples. The MAPE is used to judge the  degree of error of prediction models as the prediction result always has errors, although the prediction mode has an adequately high R2. The hit rate is derived as 
hit rate

\begin{equation} hit\ rate = \dfrac{ \sum{I\{(1 - \alpha)y_i \le \hat{y_i} \le (1 + \alpha)y_i \}} }{N} \label{eq:hitrate}\end{equation} 

其中~$0 \le \alpha \le 1$~,且~$I\{L\} = 1$~。hit rate~是用來判斷關係模型的正確率的。在本分析中,我們設定了兩個~$\alpha$~值作為~hit rate~指標用,分別是~0.1~和~0.2~,表~\ref{tab:is_result}~列出了這九個關係模型使用這三個指標配合十群交叉驗證所得到的數值,表現最好的關係模型是先使用~K-means~分群再使用~GLM~所建立的,第二好的則是先使用~Two-step~分群再使用~GLM~所建立的,圖~13~是先使用~K-means~分群再使用~GLM~所建立模型的實際~$Is$~值與使用模型得到的~$\hat{Is}$~值的比較圖,資料點的回歸取現的斜率非常接近~1~,可以看得出來兩者之間的相關度非常高。如果單看模型的線性關係表現,類神經網路的表現比~GLM~ 和線性回歸都要來的好,這也可以驗證校舍的耐震能力與其設計參數之機善一個非線性的關係,而雖然~GLM~整體的排名較~ANN~來的好,但是~hit rate~卻是~ANN~表現的比較好,但是看到~MAPE~又會發現~ANN~的~MAPE~較大,因此我們建議在~$Is$~值的關係模型的挑選,可以依據應用的需求來決定,如果需要較高準確率的時候,建議使用~ANN~,如果是需要降低整體的誤差,則建議使用~GLM~。如果先使用分群方法將校舍資料根據設計參數分出不同群集後,再對不同群集分別探勘其耐震能力與設計參數的關係模型,結果會比沒有先分群要來的好一些,探討其原因,是因為典型校舍已經是校舍建築物的一個子集合,而此子集合的特性已經非常接近,因此再進行分群也不會有顯著的改善。


\begin{table}[hbtp]
  \begin{center}
    \caption{Cross-validation result of the prediction model}
    \label{tab:is_result}
    \scriptsize
    \begin{tabular}{l c c c c c c c c c}
      \hline
       & K-means & Two-step & ANNs & K-means    & Two-step   & Regression & K-means & Two-Step & GLM \\
       &   ANNs  &   ANNs   &      & Regression & Regression &            &   GLM   &   GLM    & \\ 
      \hline
	   R              & 80.21\% & 81.78\% & 80.76\% & 72.16\% & 72.16\% & 72.15\% & 87.41\% & 87.11\% & 87.05\% \\
       MAPE           & 28.69\% & 26.64\% & 26.51\% & 46.06\% & 46.05\% & 46.05\% & 24.68\% & 24.70\% & 24.71\% \\
       hit\_rate(0.2) & 53.12\% & 54.29\% & 54.31\% & 40.16\% & 40.17\% & 40.21\% & 48.82\% & 48.64\% & 48.52\% \\
       hit\_rate(0.1) & 27.71\% & 28.72\% & 28.75\% & 21.13\% & 20.98\% & 21.09\% & 25.05\% & 25.16\% & 25.16\% \\
      \hline
       Rank & 6 & 4 & 3 & 9 & 7 & 8 & 1 & 2 & 4 \\
      \hline
      \end{tabular}
  \end{center}
\end{table}

When 0<a<1 and I{L} = 1, hit rate is utilized to determine the percentage of data predicted correctly by the prediction model, that is, prediction model accuracy. In this work, the hit rate is ranked by setting a equal to 0.1 and 0.2, which are utilized as two assessment indices that average and rank the performance of accuracy. Table 1 lists the assessment indices of the nine prediction models. The prediction model that performs best is that built using the GLM with K-means clustering. The second best prediction model is that built via the GLM with two-step clustering. Fig. 13 compares the actual CDR and CDR obtained using the K-means and GLM prediction models. The slope of the regression curve equation approaches 1, indicating a strong correlation between school building design data and aseismic ability of building. However, the scattered distribution of actual data points corresponds to a high R2 and high MAPE. After a thorough comparison of the nonlinear analysis by ANNs, the GLM, and linear analysis by simple regression, the ANNs perform better than the GLM and linear analysis by simple regression all aspects, confirming that the design parameters of school buildings have a nonlinear relationship with aseismic ability, which conforms to the fact that the aseismic CDR in this work is obtained using nonlinear analysis. Although the GLM ranks high in comprehensive assessment, its hitate is worse than that of ANNs; ANNs also have a higher MAPE. Hence, it is possible to determine which prediction method is suitable based on actual needs when predicting the aseismic ability of school buildings. We recommend using ANNs for accurate prediction of the aseismic abilities of school buildings; however, the drawback in using ANNs is that the prediction model generated is a black box. We recommend using the GLM to minimize total error. When building prediction models by clustering first and then comparing the performance of the three assessment methods, the prediction model built with clustered data performs slightly better than those built directly, indicating that traditional school buildings are already a subcluster of various architectural patterns. One feature of subclusters is their weak correlation with the aseismic ability of school buildings. Hence, information added to the cluster will not markedly improve prediction model quality.

由於校舍補強預算和時間有限,因此其執行的優先順序就會依照評估得到的耐震能力作為參考排序,因此本分析還用此實際的應用作為另一個評量關係模型優劣的指標,此指標將所有的校舍依照其~$Is$~值排序後,照順序等分成~10~群,另外在用關係模型得到的~$\hat{Is}$~排序,一樣照順序等分為 10 群,接著比較每筆校舍所分配到的群集,如果實際所屬的群集編號和關係模型得到的群集編號一樣,則誤差(error)為~0,如果差了一號,則誤差為~1,差了兩號則誤差為~2,表~\ref{tab:is_seq_result}~即為九組關係模型的排序誤差結果。可以發現表現最好的一組仍然為先使用~K-means~分群再用~GLM~ 探勘所得到的關係模型,其誤差小於 1 的資料比例為 70.8\%,誤差小於~2~的資料比例則有~88.9\%~。

The budgets and priorities for reinforcing school buildings are based on the aseismic abilities of school buildings. This work analyzed the sequencing result of aseismic ability of school buildings by sequencing school buildings based on CDR values, dividing them into 10 equal zones, and comparing the zone number of actual and predicted values. Table 2 shows the zoning result. When Error = 0, the predicted and actual values have the same zone number; when Error = 1, predicted and actual values are in adjacent zones; when Error = 2, predicted and actual values are separated by one zone. The prediction model built by the GLM with K-means clustering performs best. The zoning error of this prediction model <1 is 70.8\%, and the zoning error <2 is 88.9\%, indicating that the prediction model already has sufficient accuracy when sequencing is used.

\begin{table}[hbtp]
  \begin{center}
    \caption{Sequencing analysis of prediction of aseismic ability}
    \label{tab:is_seq_result}
    \scriptsize
    \begin{tabular}{l c c c c c c c c c}
      \hline
       Error & K-means & Two-step & ANNs & K-means    & Two-step   & Regression & K-means & Two-Step & GLM \\
             &   ANNs  &   ANNs   &      & Regression & Regression &            &   GLM   &   GLM    & \\ 
      \hline
	   0  & 32.8\% & 35.0\% & 33.9\% & 33.4\% & 34.1\% & 33.3\% & 35.6\% & 34.9\% & 35.0\% \\
	   1  & 36.0\% & 35.2\% & 36.5\% & 35.4\% & 34.2\% & 35.5\% & 35.2\% & 35.9\% & 35.7\% \\
	   2  & 17.5\% & 15.2\% & 16.2\% & 16.6\% & 17.1\% & 16.4\% & 18.1\% & 17.3\% & 17.5\% \\
	   >2 & 13.7\% & 14.6\% & 13.5\% & 14.6\% & 14.7\% & 14.8\% & 11.1\% & 11.9\% & 11.8\% \\
      \hline
      Rank & 5 & 5 & 4 & 7 & 9 & 8 & 1 & 2 & 2 \\
      \hline
      \end{tabular}
  \end{center}
\end{table}

\section{CDR 值與校舍設計之關係模型}

校舍耐震資料庫中,詳細評估表的耐震能力索引 $CDR$ 值是用來評估校舍是否需要補強、甚至是拆除的最重要依據,此數值的取得非常耗時耗力,且與校舍結構材料、設計與現況等參數之間為高度非線性的關係,如果能夠取的此一關係模型,對於校舍耐震能力補強計畫的進行,可以有很大的幫助。

\subsection{資料前處理}
\subsection{資料探勘}
\subsection{結果}


\section{D\_isR 值與校舍設計之關係模型}

「校舍是否需要補強」此一指標其數值之基礎即為詳細評估的耐震能力指標 $CDR$ 值,$CDR$ 值超過 1 表示其耐震能力尚符合安全規範,反之,則是有安全疑慮,需要進一步補強或是拆除。而 $D\_isR$ 則是標記各校設耐震能力是否足夠的二元指標,也是教育部的校舍耐震能力補強計畫中,前期篩選工作的最主要目標。找到這個指標與校舍設計、現況等參數的關係模型,與 $Is$ 值和 $CDR$ 值關係模型一樣,此一關係模型對於校舍耐震能力補強計畫中,初期的篩選工作可以有很大的助益,也可以輔助決策者編定預算、快速的根據狀況決定不同計畫年度補強的規模等。

\subsection{資料前處理}
\subsection{資料探勘}
\subsection{結果}

