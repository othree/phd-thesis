\renewcommand\thetable{\arabic{chapter}-\arabic{table}}
%\renewcommand\thefigure{\arabic{chapter}-\arabic{figure}} 
\chapter{耐震能力預測}

校舍耐震資料庫中的耐震能力索引($CDR$),是由專業人士根據校舍的設計與實際狀況建製結構模型,並進行非線性的推垮分析所得到的,它可以用數值的方式來表現該棟建築物的耐震能力,有了此一數值後,可以快速找出有安全疑慮的校舍,根據耐震能力排序,甚至推算會需要多少預算來補強多少棟校舍等等,都可以使用此數值作為依據來達成,可以說是非常重要的校舍特性,然而要取得此一數值非常耗時耗力,如果有其他快速便宜的方法可以取得此一數值,那便可以大大的減少校舍耐震能力補強作業的所需要的時間與經費,因此本研究的第一個資料探勘目標,便是找出預測耐震能力索引 $CDR$ 的預測模型,得到此一預測模型後,便可以根據校舍的設計參數與現況作為輸入參數,快速的得到該校舍的耐震能力索引參考值。



\section{Is 值預測}


\section{CDR 值預測}

