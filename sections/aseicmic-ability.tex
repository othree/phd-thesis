\renewcommand\thetable{\arabic{chapter}-\arabic{table}}
%\renewcommand\thefigure{\arabic{chapter}-\arabic{figure}} 
\chapter{耐震能力與校舍設計關係模型}

在校舍耐震資料庫中,不同階段的調查資料有不同的校舍耐震能力指標,這些指標都是以數值形式來量化校舍建築物的耐震能力,有了此一數值後,可以快速找出有安全疑慮的校舍,根據耐震能力排序,甚至推算會需要多少預算來補強多少棟校舍等等,都可以使用此數值作為依據來達成,可以說是非常重要的校舍特性,然而要取得此一數值非常耗時耗力,如果有其他快速便宜的方法可以取得此一數值,那便可以大幅度的減少校舍耐震能力補強作業的所需要的時間與經費,因此本研究的第一個資料探勘目標,便是找出預測耐震能力索引的預測模型,得到此一預測模型後,便可以根據校舍的設計參數與現況作為輸入參數,快速的得到該校舍的耐震能力索引參考值。

初步評估階段資料的耐震能力指標是 $Is$ 值,此一數值為專業人士於現場良策、調查後,使用國家地震中心的專家根據過往的實驗數據統計分析後,所設計出的一個評估表計算而得到的,為一初步的評估值,較詳細評估所計算的耐震能力指標可靠度來的低,但仍是校舍補強計畫初期的篩選工作中非常重要的數值。詳細評估和補強設計階段的耐震能力指標則是 $CDR$ 值,$CDR$ 值是由專業人士根據校舍的設計與實際狀況建製結構模型,並進行非線性的推垮分析所得到的,此數值與建築物之設計參數為高度非線性的關係,也是最接近校舍建築物實際耐震能力的量化指標。

而除了數值量化的耐震能力指標外,本研究還將「校舍是否耐震能力不足」($IsR$)這個二元指標作為耐震能力來作為校舍耐震資料庫中的第三個耐震能力指標,, $Is$、$CDR$、$IsR$ 這三個耐震能力指標的預測模型,即為本研究所取得的第一個校舍耐震資料庫隱含知識,以下便針對不同指標的預測模型的資料探勘流程與結果詳細介紹。

\section{IsR 值與校舍設計之關係模型}

「校舍是否耐震能力不足」此一指標其數值來源之基礎即為初步評估的耐震能力指標 $IsR$ 值,$Is$ 值超過 100 表示其耐震能力尚符合安全規範,反之,則是有安全疑慮,需要進一步進行詳細評估其耐震能力,考量該校舍是否需要補強,或是拆除。然而有些校舍的狀況特別,國震中心所設計的評估表可能無法真實反應其耐震能力,負責初步評估的專業人士會根據現場調查的資料以及其專業判斷是否要以特別案例提出,這些校舍的 $Is$ 值可能有達到 100,但是卻還是會被列在校舍是否耐震能力不足的名單內,

\section{Is 值與校舍設計之關係模型}


\section{CDR 值與校舍設計之關係模型}

