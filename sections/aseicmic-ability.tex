\renewcommand\thetable{\arabic{chapter}-\arabic{table}}
%\renewcommand\thefigure{\arabic{chapter}-\arabic{figure}} 
\chapter{校舍資訊與耐震能力之關係模型}

在校舍耐震資料庫中,不同階段的調查資料有不同的校舍耐震能力指標,這些指標都是以數值形式來量化校舍建築物的耐震能力,有了此一數值後,可以快速找出有安全疑慮的校舍,根據耐震能力排序,甚至推算會需要多少預算來補強多少棟校舍等等,都可以使用此數值作為依據來達成,可以說是非常重要的校舍特性,然而要取得此一數值非常耗時耗力,如果有其他快速便宜的方法可以取得此一數值,那便可以大幅度的減少校舍耐震能力補強作業的所需要的時間與經費,因此本研究的第一個資料探勘目標,便是找出預測耐震能力索引的預測模型,得到此一預測模型後,便可以根據校舍的設計參數與現況作為輸入參數,快速的得到該校舍的耐震能力索引參考值。

初步評估階段資料的耐震能力指標是 $Is$ 值,此一數值為專業人士於現場良策、調查後,使用國家地震中心的專家根據過往的實驗數據統計分析後,所設計出的一個評估表計算而得到的,為一初步的評估值,較詳細評估所計算的耐震能力指標可靠度來的低,但仍是校舍補強計畫初期的篩選工作中非常重要的數值。詳細評估和補強設計階段的耐震能力指標則是 $CDR$ 值,$CDR$ 值是由專業人士根據校舍的設計與實際狀況建製結構模型,並進行非線性的推垮分析所得到的,此數值與建築物之設計參數為高度非線性的關係,也是最接近校舍建築物實際耐震能力的量化指標。

而除了數值量化的耐震能力指標外,本研究還將「校舍是否需要補強」($D\_isR$)這個二元指標作為耐震能力來作為校舍耐震資料庫中的第三個耐震能力指標,$Is$、$CDR$、$D\_isR$ 這三個耐震能力指標的預測模型,即為本研究所取得的第一個校舍耐震資料庫隱含知識,以下便針對不同指標的預測模型的資料探勘流程與結果詳細介紹。

\section{Is 值與校舍設計之關係模型}

校舍耐震資料庫中的初步評估資料表的耐震能力索引是為 $Is$ 值,初步評估是校舍耐震能力補強作業中,很重要的篩選過程,因為建築物的詳細評估所需要的金額極高,評估需要的時間也很長久,因此需要一個速度、價錢和可靠度都可以接受的評估方法,用比較短的時間找出耐震能力有疑慮的校舍,盡早處理。而地震中心所設計的初步評估表即為一個速度、價錢和可靠度都可以接受的評估方法,只要根據建築物的設計參數,像是樓層數、長度、深度、柱子尺寸以及校舍現況等,就可以快速的得到一個極具參考價值的耐震能力指標,而其計算的原理也與詳細評估的耐震能力指標索引 $CDR$ 值一樣,為耐震容量與耐震需求的比。找出 $Is$ 值與校舍設計參數之間的關係模型,可以進一步分析初步評估表中各種數據的重要性,並將結果回饋到此一初步評估表,將初步評估表調整的更為精簡,負責評估的人員也可以更快速的完成此意評估表格。

The proposed prediction model, based on data-mining technology, has six steps – business understanding, data understanding, data preparation, modeling, evaluation, and deployment – as recommended in the cross industry standard process for data mining (CRISP-DM) proposed by Chapman et al. (2000) (Fig. 9). The objective of this work is to construct a prediction model for the aseismic ability indices of school buildings and collect complete data. Hence, the major tasks are data preparation, modeling, and evaluation. Fig. 10 shows the procedural plan based on needs and the methods chosen. 

\subsection{資料前處理}

首先,從國震中心的校舍耐震資料庫中,將各棟校舍的 $Is$ 值與其校舍的相關資料挑出並整合在一起,第一步是過濾掉明顯不合理的錯誤資料,這些資料通常是人為的錯誤造成的,而過濾的方法則是使用國陣中心建議的過濾條件:

\begin{itemize}
\item 校舍總長或總深度超過兩百公尺
\item 有任意牆厚度超過五十公分
\item 有任意柱的長或寬超過一百公分
\item 柱間垮距超過八公尺或小於兩公尺
\end{itemize}

大約有八百筆資料符合這組基本的過濾條件,而在初步的過濾之後,前處理的第二個步驟則是減少資料屬性的維度,主成分分析(Principal component analysis)是很常使用的資料前處理方法,這個分析方法可以找出所有資料屬性中,重要度較高的屬性,它是把原始的多個資料屬性,透過向量轉換的方式,線性組合出主要的資料屬性。

Roughly 800 bits of valid data are obtained. After filtering using these conditions. Principal component analysis (PCA) is then used to reduce the dimensionality of data attributes. Notably, PCA, a very common data preparation method, can identify very important attributes among various attributes. The goal is to convert the original variables through vector transition into mutually independent variables of a linear combination. The ideal situation is that principal components obtained from linear combination retain most of the information of original variables.

校舍的分群則是基於校舍的設計模式,目標是要找出校舍建築物的幾種特定設計模式,也就是要分析校舍的幾何設計參數、柱量、牆量等等,經過不斷的測試及調整,最後找出了五個主要成分屬性:

\begin{itemize}
\item 走廊柱資訊
\item 教室柱資訊
\item 強設計資訊
\item 牆量資訊
\item 柱量資訊
\end{itemize}


Clustering analysis of school buildings is based on the design patterns of school buildings; the goal is to find several design patterns. Hence, the attributes for analysis are the geometric information of school buildings, such as dimensions and quantity of walls and columns, and width and height of school buildings. Other attributes, such as year of construction and locale, may not be incorporated into PCA analysis. After continuous testing and adjustment, five major attributes are obtained and the importance of constitutional fields is considered as the basis for naming. The five attributes are as follows: corridor column information; classroom column design information; wall design information; data for number of walls; and, data for number of classroom columns.



在前面的處理完成後,還使用分群分析將校舍分群,以幫助後續的 $Is$ 值關係模型的建立,分群的目標是將校舍依照不同的設計模式特性區分開來,將相近設計的校舍放在同一群集,分群的依據是主成分分析法所找出的五種主要校舍設計相關屬性,圖十一是使用 SPSS Clemitine 進行此一分群分析時的節點設計圖,分別使用的 K-means 和 Two-Step 兩種分群演算法,其參數設定如下:

After data preparation, clustering analysis is utilized to find hidden design patterns of school buildings. This study uses the K-means and two-step clustering methods. Fig. 11 shows the node deployment for clustering using the SPSS Clementine (2007) software. The parameter choices for the two methods are as follows.

\subsubsection{K-means}

此一分群方法最重要的設定參數是初始的群集數 $k$,雖然有許多研究都在研究如何找出最佳的 $k$ 值,但是目前仍沒有一個方法可以宣稱它找到 $k$ 值是最佳的,唯有對該領域的專業了解以及詳細的測試才能得到最佳的 $k$ 值,在本分析案例中,K-means 分群的停止條件是設定為 20 次迭代,如果 $k$ 太大,那會造成分群結果無法在 20 次迭代內收斂,如果 $k$ 小於 6,則分群的結果就可以在 20 次迭代內完整的收斂,每筆資料都會故底定在所屬的群集內,不在變動,最後挑選的 $k$ 值是 5,而根據此意參數的分群結果可以找出三個主要的校舍群集,分別的比例是 28\%、56\% 和 16\%。

The most important parameter with this clustering method is the initial group, k. Although many researchers have developed methods for choosing the initial K value, no method can confirm that it has found the best K value. The best K can be found only based on researcher understanding and problem testing. In this study, K-means clustering is set to stop after 20 iterations. If the K value is too large and cannot be completely converged after 20 operations, some data points will continue to change the clusters to which they belong. When the K value is reduced to <6, Clustering models can be completely converged after 20 iterations, become stable, and no longer change cluster label of all data bit. The K value chosen is 5, and three major clusters are obtained. The remaining two clusters have few data and are considered outliers. The distribution of three major clusters are 28\%, 56\%, and 16\%.

\subsubsection{Two-Step}

Two-Step 分群有兩個優點,一是複雜度不高,運算時間與資料數量間之關係為線性關係,第二個優點就是不需要由人工決定分群的群數,演算法即可自己根據資料狀況決定,操作人員只需給予上下限,在本分析中,上下限的設定為最少兩個群集,最多八個群集,而最後的分析結果是所有的資料都被分到兩個群集中,分別佔了 54\% 和 45\%。

This clustering method has two features. The first is enhanced scalability. The algorithm has low complexity. Computing time does not grow nonlinearly as data volume increases. The other feature is that it can determine the number of clusters, unlike K-means clustering, which requires manual designation of parameters. However, this work can designate the upper and lower limits for the number of clusters. This work sets the limit to 2–8 clusters based on experience with K-means clustering. Consequently, all data are divided into two clusters, accounting for 54\% and 45\% of all data.






此資料探勘分析還用了十群交叉驗證來驗證結果的可靠度,因此資料前處理的最後一個步驟就是將整理好的資料隨機分為十組。

This work uses 10-fold cross validation to validate the prediction model. After preparation, data are grouped by first dividing data randomly and equally into 10 clusters. One cluster is then chosen as a dataset for validation and the remaining nine clusters are combined into one training dataset.

\subsection{資料探勘}

完成資料前處理,將校舍的群集分好之後,才開始建立 $Is$ 值與校舍設計參數的關係模型,本分析使用了廣義線性模型、線性回歸和類神經網路三種分析方法,每種方法都有三種分析資料群,分別為先經過 K-means 分群的資料、先經過 Two-step 分群的資料及沒有先經過分群的資料。圖 12 為使用 SPSS Clemitine 分析時的節點設計圖。以下分別對三種分析方法的參數設定作說明。

The prediction model is built only after clustering is completed. This work uses a GLM, simple regression, and ANNs to build the prediction model based on three groups of data – not clustered in advance, clustered by K-means, and clustered by the two-step method in advance. In total, nine prediction models are generated. Fig. 12 shows the node configuration within SPSS Clementine. Below are the parameters chosen for the three methods. 

\subsubsection{Generalized linear model}

廣義線性模型是假設在輸入參數和預測目邊之間有一個可以用連結函數表達的關係,這個連結函數可能是指數函數、對數函數、Logistic 函數等,經由一些測試資料的測試,我們選擇使用對數函數作為連結函數,並且根據實際的資料分布選擇了常態分布作為輸入參數的分布函數,而關係模型的分布也選擇常態分布,因其表現教其他分布形式較好。

The GLM assumes a relationship between input variables and a predictor; this relationship can be built by a link function such as identity function, log function, logit function, or power function. After available link functions testing on some data, the prediction model performs best when using log function as the link function; hence, this work chose the log function. The distribution function of the predictor is based on the actual distribution of data. This work chose the normal distribution, which is close to the real data distribution. The prediction model constructed using the normal distribution performed better than those with other distribution types.

\subsubsection{Simple regression}

線性回歸是選擇使用最小平方根法來建立校舍建築物的設計參數與其耐震能力 $Is$ 之間的關係,這也是最常使用的回歸方法之一。

Simple regression in this work uses the least square method by adopting the building design parameters as independent variables $X$ and the aseismic ability of buildings as dependant variables $Y$. The linear equation between regressed design parameters and aseismic indices serve as the model for predicting aseismic ability $Y$ based on building design parameters $X$.

\subsubsection{Artificial neural networks}

類神經網路需要決定的參數包括隱藏層的數量、每層的神經元數量、學習率、停止條件等,除了直接設定神經網路的參數,還有一些方法可以使用,例如 dynamic、multiple 或是 prune method 可以用來調整並找出最佳的神經網路大小和結構,dynamic method 是從一個小型的神經網路開始(兩個隱藏層、每層兩個神經元),慢慢成長,並且比較成長前後的神經網路效能與結果,Multiple method 則是同時產生各種不同的神經網路,並且一起訓練到達停止條件,然後在從中挑選出表現最好的一個,而 Prune Method 則是從一個大的類神經網路開始,慢慢的把重要度低的神經元節點拿掉。本分析最後挑選的是 Exhaustive Prune Method,是 Prune Method 的一種修改形式,對於節點的篩選要求較高,是所有方法中最花時間的,但是通常也可以找到最好的結果。其他的類神經網路設定參數為:初始的神經網路為兩層隱藏層,其中一層有 30 個神經元、一層有 20 個神經元,停止條件為 250 個訓練循環,在這個設定下,Exhaustive Prune Method 是表現最好的方法。

Generally, ANNs must decide on such parameters as number of hidden layers, number of neurons in each layer, learning rate, and stop condition. Aside from directly setting these parameters, methods such as dynamic, multiple, and Prune methods are available for adjusting and finding the optimal size and structure of the neural network. The dynamic method starts with a small neural network (two hidden layers with two neurons for each layer), expands network size gradually, and decides on the further expansion based on model performance before and after expansion. The multiple methods constructs multiple neural networks simultaneously, trains all neural networks to reach the ``stop condition,'' and then selects the group with the best performance. In contrast with the dynamic method, which slowly builds a large neural network from a small one, the Prune method first builds a large network and then removes neurons with low importance based on training. This work chooses the exhaustive Prune method, a special application of the Prune method. The initial neural network has two hidden layers, one with 30 neurons and the other with 20 neurons. The stop condition is set to 250 training cycles. Under this limit, the prediction model built by the exhaustive Prune method performs best.


\subsection{驗證}

驗證有兩個主要的目的,一是確保資料探勘找到的關係模型的可靠度,而不會找到只適用於該組訓練資料集的關係模型,第二個目的是可以用來作為比較不同分析方法的指標數據,本分析使用的驗證方式是十群交叉驗證,這個方法將所有的資料等分成十份,每次挑選九組出來作為訓練資料集,留下一組作為驗證資料集,如此可以得到十組模型以及其可靠度的指標,求此十組指標平均值即可得到代表此關係模型的可靠度代表值,而本分析所選擇的指標有三個,線性關係、絕對平均誤差(Mean Absolute Prediction Error, MAPE)以及 hit rate。

Validation work has two purposes. The first is to ensure model reliability instead of to generate only good performance during data training. The second is to serve as a benchmark for comparing the performance of different prediction models. This work uses 10-fold cross validation to assess and compare the performance of prediction models. This method divides a fixed amount of data into 10 groups, conducts 10 rounds of model building and validation, chooses a different group of data for testing, trains the model with remaining nine groups of data, and uses test group data to validate model accuracy. After validation for 10 times, the accuracy of the 10 models is obtained and their average is taken as the accuracy of this algorithm. This study uses linear correlation, mean absolute prediction error (MAPE), and hit rate as the indices for comparing the prediction model performance.

\subsection{結果}

此資料探勘分析會產生九個校舍耐震能力與校舍建築設計資訊間的關係模型,包括直接使用GLM、線性回歸和類神經網路的三組,混合使用 K-means 或是 Two-Step 分群的六組,要比較其優劣我們使用了線性關係 $R$ 、 MAPE 和 hit rate 三個指標,並配合十群交叉驗證,其中線性關係 $R$ 之公式為:

This work constructed nine prediction models; three are directly generated by the GLM, simple regression, and ANNs. The mixed model of K-means and two-step clustering generated three prediction models. Hence, nine models were obtained and 10-fold cross validation is used to compare the performance of the three reference indices – R2, MAPE, and hit rate. Notably, R2, the linear correlation is

\begin{equation} R = \dfrac{\sum{(\hat{y_i} - \tilde{y})^2}}{\sum{(y_i - \tilde{y})^2}} \label{eq:RSQ}\end{equation} 

其中 $y_i$ 是校舍的實際耐震能力指標 $Is$ 值, $\hat{y_i}$ 則是透過此關係模型的到的推估耐震能力指標值,$\tilde{y}$ 則是所有資料的 $Is$ 值之平均,透過此一公式即可得到實際的 $Is$ 值與透過關係模型得到的推估值之間的線性關係,線性關係越高表示兩者之間越接近,也代表著關係模型的正確性。MAPE 的公式為:

where yi is the aseismic CDR of school buildings obtained using nonlinear analysis of the database, yi is the CDR obtained from the prediction model, and y~ is the average aseismic CDR of school 651 buildings obtained using nonlinear analysis. The correlation between aseismic CDR of school buildings obtained via the prediction model and nonlinear analysis can be determined based on linear correlation. A high CDR indicates a strong correlation and many opportunities to make correct predictions. The MAPE is derived as

\begin{equation} MAPE = \dfrac{\sum{\dfrac{y_i - \hat{y_i}}{y_i}}}{N} \label{eq:MAPE}\end{equation} 

其中 $N$ 是資料總數,MAPE 的是用來表示關係模型誤差之數值,由於關係模型不可能完全沒有誤差,即使有非常高的線性關係也是會有誤差,因此會使用 MAPE 作為判斷其誤差程度的參考。 hit rate 的定義是:

where N is the number of samples. The MAPE is used to judge the  degree of error of prediction models as the prediction result always has errors, although the prediction mode has an adequately high R2. The hit rate is derived as 
hit rate

\begin{equation} hit\ rate = \dfrac{ \sum{I\{(1 - \alpha)y_i \le \hat{y_i} \le (1 + \alpha)y_i \}} }{N} \label{eq:hitrate}\end{equation} 

其中~$0 \le \alpha \le 1$~,且~$I\{L\} = 1$~。hit rate~是用來判斷關係模型的正確率的。在本分析中,我們設定了兩個~$\alpha$~值作為~hit rate~指標用,分別是~0.1~和~0.2~,表~\ref{tab:is_result}~列出了這九個關係模型使用這三個指標配合十群交叉驗證所得到的數值,表現最好的關係模型是先使用~K-means~分群再使用~GLM~所建立的,第二好的則是先使用~Two-step~分群再使用~GLM~所建立的,圖~13~是先使用~K-means~分群再使用~GLM~所建立模型的實際~$Is$~值與使用模型得到的~$\hat{Is}$~值的比較圖,資料點的回歸取現的斜率非常接近~1~,可以看得出來兩者之間的相關度非常高。如果單看模型的線性關係表現,類神經網路的表現比~GLM~ 和線性回歸都要來的好,這也可以驗證校舍的耐震能力與其設計參數之機善一個非線性的關係,而雖然~GLM~整體的排名較~ANN~來的好,但是~hit rate~卻是~ANN~表現的比較好,但是看到~MAPE~又會發現~ANN~的~MAPE~較大,因此我們建議在~$Is$~值的關係模型的挑選,可以依據應用的需求來決定,如果需要較高準確率的時候,建議使用~ANN~,如果是需要降低整體的誤差,則建議使用~GLM~。如果先使用分群方法將校舍資料根據設計參數分出不同群集後,再對不同群集分別探勘其耐震能力與設計參數的關係模型,結果會比沒有先分群要來的好一些,探討其原因,是因為典型校舍已經是校舍建築物的一個子集合,而此子集合的特性已經非常接近,因此再進行分群也不會有顯著的改善。


\begin{table}[hbtp]
  \begin{center}
    \caption{Cross-validation result of the prediction model}
    \label{tab:is_result}
    \scriptsize
    \begin{tabular}{l c c c c c c c c c}
      \hline
       & K-means & Two-step & ANNs & K-means    & Two-step   & Regression & K-means & Two-Step & GLM \\
       &   ANNs  &   ANNs   &      & Regression & Regression &            &   GLM   &   GLM    & \\ 
      \hline
	   R              & 80.21\% & 81.78\% & 80.76\% & 72.16\% & 72.16\% & 72.15\% & 87.41\% & 87.11\% & 87.05\% \\
       MAPE           & 28.69\% & 26.64\% & 26.51\% & 46.06\% & 46.05\% & 46.05\% & 24.68\% & 24.70\% & 24.71\% \\
       hit\_rate(0.2) & 53.12\% & 54.29\% & 54.31\% & 40.16\% & 40.17\% & 40.21\% & 48.82\% & 48.64\% & 48.52\% \\
       hit\_rate(0.1) & 27.71\% & 28.72\% & 28.75\% & 21.13\% & 20.98\% & 21.09\% & 25.05\% & 25.16\% & 25.16\% \\
      \hline
       Rank & 6 & 4 & 3 & 9 & 7 & 8 & 1 & 2 & 4 \\
      \hline
      \end{tabular}
  \end{center}
\end{table}

When 0<a<1 and I{L} = 1, hit rate is utilized to determine the percentage of data predicted correctly by the prediction model, that is, prediction model accuracy. In this work, the hit rate is ranked by setting a equal to 0.1 and 0.2, which are utilized as two assessment indices that average and rank the performance of accuracy. Table 1 lists the assessment indices of the nine prediction models. The prediction model that performs best is that built using the GLM with K-means clustering. The second best prediction model is that built via the GLM with two-step clustering. Fig. 13 compares the actual CDR and CDR obtained using the K-means and GLM prediction models. The slope of the regression curve equation approaches 1, indicating a strong correlation between school building design data and aseismic ability of building. However, the scattered distribution of actual data points corresponds to a high R2 and high MAPE. After a thorough comparison of the nonlinear analysis by ANNs, the GLM, and linear analysis by simple regression, the ANNs perform better than the GLM and linear analysis by simple regression all aspects, confirming that the design parameters of school buildings have a nonlinear relationship with aseismic ability, which conforms to the fact that the aseismic CDR in this work is obtained using nonlinear analysis. Although the GLM ranks high in comprehensive assessment, its hitate is worse than that of ANNs; ANNs also have a higher MAPE. Hence, it is possible to determine which prediction method is suitable based on actual needs when predicting the aseismic ability of school buildings. We recommend using ANNs for accurate prediction of the aseismic abilities of school buildings; however, the drawback in using ANNs is that the prediction model generated is a black box. We recommend using the GLM to minimize total error. When building prediction models by clustering first and then comparing the performance of the three assessment methods, the prediction model built with clustered data performs slightly better than those built directly, indicating that traditional school buildings are already a subcluster of various architectural patterns. One feature of subclusters is their weak correlation with the aseismic ability of school buildings. Hence, information added to the cluster will not markedly improve prediction model quality.

由於校舍補強預算和時間有限,因此其執行的優先順序就會依照評估得到的耐震能力作為參考排序,因此本分析還用此實際的應用作為另一個評量關係模型優劣的指標,此指標將所有的校舍依照其~$Is$~值排序後,照順序等分成~10~群,另外在用關係模型得到的~$\hat{Is}$~排序,一樣照順序等分為 10 群,接著比較每筆校舍所分配到的群集,如果實際所屬的群集編號和關係模型得到的群集編號一樣,則誤差(error)為~0,如果差了一號,則誤差為~1,差了兩號則誤差為~2,表~\ref{tab:is_seq_result}~即為九組關係模型的排序誤差結果。可以發現表現最好的一組仍然為先使用~K-means~分群再用~GLM~ 探勘所得到的關係模型,其誤差小於 1 的資料比例為 70.8\%,誤差小於~2~的資料比例則有~88.9\%~。

The budgets and priorities for reinforcing school buildings are based on the aseismic abilities of school buildings. This work analyzed the sequencing result of aseismic ability of school buildings by sequencing school buildings based on CDR values, dividing them into 10 equal zones, and comparing the zone number of actual and predicted values. Table 2 shows the zoning result. When Error = 0, the predicted and actual values have the same zone number; when Error = 1, predicted and actual values are in adjacent zones; when Error = 2, predicted and actual values are separated by one zone. The prediction model built by the GLM with K-means clustering performs best. The zoning error of this prediction model <1 is 70.8\%, and the zoning error <2 is 88.9\%, indicating that the prediction model already has sufficient accuracy when sequencing is used.

\begin{table}[hbtp]
  \begin{center}
    \caption{Sequencing analysis of prediction of aseismic ability}
    \label{tab:is_seq_result}
    \scriptsize
    \begin{tabular}{l c c c c c c c c c}
      \hline
       Error & K-means & Two-step & ANNs & K-means    & Two-step   & Regression & K-means & Two-Step & GLM \\
             &   ANNs  &   ANNs   &      & Regression & Regression &            &   GLM   &   GLM    & \\ 
      \hline
	   0  & 32.8\% & 35.0\% & 33.9\% & 33.4\% & 34.1\% & 33.3\% & 35.6\% & 34.9\% & 35.0\% \\
	   1  & 36.0\% & 35.2\% & 36.5\% & 35.4\% & 34.2\% & 35.5\% & 35.2\% & 35.9\% & 35.7\% \\
	   2  & 17.5\% & 15.2\% & 16.2\% & 16.6\% & 17.1\% & 16.4\% & 18.1\% & 17.3\% & 17.5\% \\
	   >2 & 13.7\% & 14.6\% & 13.5\% & 14.6\% & 14.7\% & 14.8\% & 11.1\% & 11.9\% & 11.8\% \\
      \hline
      Rank & 5 & 5 & 4 & 7 & 9 & 8 & 1 & 2 & 2 \\
      \hline
      \end{tabular}
  \end{center}
\end{table}

\section{CDR 值與校舍設計之關係模型}

校舍耐震資料庫中,詳細評估表的耐震能力索引 $CDR$ 值是用來評估校舍是否需要補強、甚至是拆除的最重要依據,此數值的取得非常耗時耗力,且與校舍結構材料、設計與現況等參數之間為高度非線性的關係,如果能夠取的此一關係模型,對於校舍耐震能力補強計畫的進行,可以有很大的幫助。

\subsection{資料前處理}

Manually inputting data may result in incorrect units or formats because controlling data quality in the real world is difficult. Hence, it is necessary to pre-process data before building the relational model. Quality, as an important part of soft computing and data analysis, has considerable influence on subsequent analytic results or even on the reliability of the generated model. Apart from the actual data, the researcher also refers to expert advice from NCREE for data pre-processing. The main target of data pre-processing is to ensure the accuracy and adjustment of the data in a format that clearly reflects the target of analysis. Pre-processing includes data screening, property screening, and new property synthesis. Data screening is divided into two stages. The first stage is the rationality screening of the data. Pre-existing mistakes are unavoidable because the data in the School Building Database were entered manually; the obligation is to identify such mistakes. Most of the school buildings were I-type shaped buildings; a very common design. . With regard to the properties of these school buildings, the NCREE (2005) suggests the following screening conditions:

\begin{itemize}
\item Total depth of the school building should exceed 20 meters or is less than 6 meters
\item The span exceeds 8 meters or is less than 2 meters
\item The number of spans for a single classroom is less than 1
\item The number of columns in the classroom is low
\item The collapse ground acceleration of the major direction is greater than that of the minor direction
\end{itemize}

In the second stage, choosing school buildings with both basic design parameters and minimum destruction ground acceleration is necessary because not all school buildings have detailed information.According to the raw data in the seismicassessment database for school buildings, each data set contains hundreds of properties. Based on our judgment with expert which are non-structural and low importance, and synthesize some properties with similarity. There are still more than 30 properties left after this reduction process. This study further classifies school building records into subsets based on similarities in property values, and chooses one subset with major population as the data set for further studying. After the classification of school buildings, we try to do further reduction and finally determine a set of key properties which is optimal to represent the seismic characteristics of individual school buildings. The choice is based on data distribution, and a subset that correctly represents I-shaped school buildings. The features of this subset adopted in this study are listed below:

\begin{itemize}
\item No corridor columns
\item Only use one type of classroom column
\item School buildings have no RC walls
\item School buildings have no brick walls with four-side confinement
\item School buildings have no brick walls with three-side confinement
\end{itemize}

After finalizing the data for the first and second stages, 107 datasets conform to the above condition. Twelve properties are then chosen for the screened data with reference to expert advice, displayed as P1 to P12 in Table 1. In addition to the screening based on existing properties, this study synthesizes two new properties, P13 and P14. They represent the number of classrooms and number of spans for a single classroom, respectively, based on expert advice and existing data. In P11, SDS stands for design spectral response accelerations at short-periods, and in P12, SD1 stands for design spectral response accelerations at 1 sec. These two parameters represent the magnitude of the seismic force at the building’s location, and are very important parameters for analyzing the aseismic ability of buildings by non-linear analysis.

The last step of data preprocessing is normalization. The purpose of normalization is to balance the impacts of the parameters in different scales. If an input parameter has small values of mean and standard deviation, but is of high importance and if the result is also sensitive to this parameter, then it is necessary to use data normalization to prevent its influence from being overshadowed by other larger scale parameters. Normalization methods include converting the data into the range of 0 to 1, using the maximum and minimum values, and converting data to the standard deviation of its mean. The normalization principle adopted in this paper is to retain the original values as far as possible, so only a few parameters with large values, such as P2, P3, P7, P9 and P10, are divided by 1000 to make their scale comparable to other parameters.

After the three pre-processing steps, 107 datasets and 14 properties were obtained. The subsequent analysis was based on this dataset.

\subsection{資料探勘}

\subsubsection{Genetic Programming}

An AC model was built for this study to represent the relationship between the basic design parameters of school buildings, and minimum destruction ground acceleration. GP was the first model to be used, and based on a preset number of tiers for different operation trees; it can result in relational equations with different degrees of complexity. In this case, several operation trees with different number of tiers were tested, and it was found that the most suitable number is either four or five. Having a low number of tiers leads to reduced complexity of the relation model and hence, poor performance. Conversely, large numbers result in many difficulties, such as convergence problems, time-consuming progressive computation, and a very complicated relationship model. The optimum setting is 200 populations of 5000 progressive iterations, a crossover rate of 0.8, and a mutation rate of 0.1. The crossover function used in this paper is the scattered function. The mutation function is the adaptive feasible function. Both these functions can be applied to solve many different problems. The scattered function diversifies the child layer after crossover. The adaptive feasible function is suitable for the constrained minimization problem. This setting was chosen after the analysis was conducted 30 times. Table 2 shows the root mean square (RMSE) of the model generated.

RMSE is the index used in the current study to judge the quality of models, and is defined as the equation below:

\begin{equation} RMSE = \sqrt{\dfrac{\sum{(y_i - \hat{y_i})^2}}{N}} \label{eq:RMSE}\end{equation}

where n is the number of datasets, $y$ is the estimated value obtained from the equation, and $\hat{y}$ is the actual unit-less deviation index value (the smaller the better). The ground acceleration of the minimum destruction obtained from the nonlinear analysis was distributed between 0.04 and 0.5, and therefore, the relationship model has a sufficient recognition rate. Experts from NCREE recommend that the RMSE must be below 0.04. The control group is the relationship model obtained from the artificial neural network. In applying the relationship model constructed by SPSS Clementine and choosing the Exhaustive Prune method to adjust the number of tiers and nodes, the initial neural network has two hidden tiers with 30 and 20 neurons, respectively. The neurons have been trained for 250 iterations, and those with a low degree of importance are removed during the training period based on the situation. The resulting RMSE is 0.041, which is close to the target of 0.04. In this paper we use WGP to create an aseismic ability prediction model for real school buildings. The quality of our model is similar to models built using Artificial Neural Networks. However, Artificial Neural Network based models are complicated; their mechanism is in a black box. The WGP model, on the other hand, is just an equation of the building’s design parameters and its aseismic ability. Thus, it can easily be ported to other platforms and programming languages for use in many applications. The optimum model obtained from the GP pattern is represented by the relationship equation below. Function nodes in a tree topology, displayed in Figure 8, uses several symbols and text to represent the F of that node.

"+" represents $f = x_1 + x_2$;
"-" represents $f = x_1 - x_2$;
"×" represents $f = x_1 \times x_2$;
"÷" represents $f = x_1 / x_2$; and
"pow" represents the power function, $f = {x_1} ^ {x_2}$.

The performance of this model is not ideal because the RMSE can only reach 0.056. Based on Figure 9, the relationship model generated did not correctly build the relationship between the design parameters of school buildings, and ground acceleration of the minimum destruction. As only three input parameters were used, it resulted in the minimum equation as the linear equation. This can be attributed to the fact that this seismic ability model for school buildings has high complexity, and the application of GP pattern alone cannot obtain the relationship between them.

The WGP pattern was then used to build the relationship equation between the design parameters of school buildings, and ground acceleration of the minimum destruction because it could be used for relationships that are more complex than with the GP pattern.

\begin{equation} AC = \dfrac{P_3}{P_{11}} + P_3 P_{12} - P_3 {P_{12}}^3  \label{eq:GP_AC}\end{equation}


\subsubsection{Weighted Genetic Programming}

WGP was used as the second model, which also chose five tiers of the operation tree. The optimum setting of GA is the same with GP: 200 populations of 5000 progressive iterations, a crossover rate of 0.8, and a mutation rate of 0.1. The best group was chosen after the analysis was conducted 30 times. Contrary to GP, the weight was set from +10 to -10 within the weighted (w) range. Table 2 shows the RMSE of the model generated, and the four-tier optimum equation is shown as Equation (9). The tree topology generated is displayed in Figure 10, and uses the same symbol as the GP tree topology to represent the same operator. The other symbol that was used is a black solid dot, which represents f=w1x1.

\begin{equation} AC = {({165 P_{10}}^{8.86 P_6} {P_4}^{4.86} + \dfrac{22.5 P_8 + 39.5 P_{10}}{P_{10}} )}^{-98.6 P_4{P_8}^{-1.3} - 133 P_{10} - 0.05 {P_7}^{1.38 P_{10}} }  \label{eq:WGP_AC}\end{equation}


Table 2 shows the RMSE of the optimum relationship equations generated from the two patterns. The RMSE reached 0.039, which is better than the performance of the model built by the artificial neural networks in the contrast group. Figure 12 displays the comparison between the estimated, and the actual values of the model. By comparing the results, the model constructed by the WGP pattern is superior to the model constructed by the GP pattern.

The parameters entered are analyzed based on the equation obtained from WGP, and Table 3 shows the input parameters obtained from the optimum relationship equations. SDS and SD1 were not used, as both are relevant to the demand of the CDR. For this study, the target is to estimate the Capacity (Aseismic Ability Index), which is irrelevant to Demand. Hence, this result conforms to the expectations. As all of the remaining input parameters were used, this indicates that the parameters chosen at the data processing stage are important.

\subsubsection{Capacity Index Formulation Tuning}

Similar with the CDR obtained from the detailed estimation, the aseismic capacity index of school buildings is the demand ratio of the aseismic capacity, excluding the ability unit and measure. CDR is directly compared to the ground acceleration, and IS is the estimated force ratio. If CDR is greater than 1, then the building has sufficient aseismic capacity. If CDR is 1, the building has an aseismic capacity equal to the demand. However, it should be noted that IS is a hundred-mark system, and 100 indicates that the capacity is equal to the demand. IS also needs to consider the usage coefficient I, of buildings. When IS equals 1.25, CDR is 1 and IS is 80. This relationship can be described by a formula that converts IS into CE, which has the same meaning as the ground acceleration of the minimum destruction.

\begin{equation} C_E = f(Is) = \dfrac{Is \times Demand \times I}{80}  \label{eq:CE}\end{equation}

I is the usage coefficient that intends to keep the seismic design at the conservative side, and prevents miscalculations caused by the insufficient estimation of the earthquake destruction. For this study, I is set to a constant of 1.25 for school buildings. Demand is set to different recommended values based on the positions of the school buildings. It represents the minimum ground acceleration that the buildings can withstand in the area. The School Building Database contains the demand data that was determined by engineers, based on the actual situations. This study adds T to CE as a revised formula, such that it is closer to the minimum destruction ground acceleration AC, of school buildings obtained from nonlinear analysis.

\begin{equation} AC = C_E + T  \label{eq:AC}\end{equation}

Based on the analysis above, this study only uses the WGP pattern to build the model, and this can be applied to complicated situations. Parameters chosen are: the operation tree has four to five tiers, 200 populations of 5000 progressive iterations, crossover rate of 0.8, mutation rate of 0.1, and choosing the best group after the analysis was conducted 30 times. The optimum T generated from the four-tier operation tree is shown in the equation below, and the tree topology is displayed in Figure 11.

\begin{equation} AC =  C_E + 0.54 \dfrac{P_3}{P_2} - 24.86 \dfrac{ {P_6}^{0.36P_3} }{ {P_5}^{2.6P_9} } - ( 37.84 \dfrac{P_5}{0.67P_2 + 5.52P_7} )^{\left[ (-0.67 P_{12})^{1.23P_7/P_{13}} \right]} \label{eq:WGP_AC_IS}\end{equation}


\subsection{結果}

Table 2 shows the RMSE after the revision. The RMSE of the IS formula used by NCREE in the preliminary appraisal is 0.067. Although there is still a gap between the target of 0.04 and this value, the emphasis is on the degree of relationship between them, and the main target is to screen out the school buildings with degrees of higher risk. Figure 13 shows the comparison between the CE converted from IS of NCREE, and the destructive ground acceleration obtained through linear analysis. Even though the data have the correct directional tendency, the model obtained from the GP method is better because of higher deviation, and a linear relationship. Subsequent to adding T to the revised formula, RMSE is reduced to 0.045. Figure 13 shows the comparison, and it can be seen that the directional tendency is quite close, thus reducing the deviation. A good revision effect is observed, making the screening result of the preliminary appraisal more accurate.

Table 3 shows the input parameters used by the equation obtained from the WGP method, and by the optimum relationship equations with four to five tiers. The number of floors (P1), total floorage (P10), and number of spans in a single classroom (P13) were not used because the number of floors and total floorage, which are highly important, were already used in the IS formula. Hence, IS has been correctly included in the two properties above, and the number of spans in a single classroom can be inferred as hidden in the formula. Hence, T in the revised formula will not use this input parameter. However, the input parameter is still used to directly construct the relationship equation in the previous case, and has a certain degree of importance.

In order to estimate the aseismic ability of a building, we need to obtained detailed information about its geometric dimensions, the properties of materials used, etc. and we may need to perform sampling and other experiments to obtain certain information e.g. the properties of materials used. Based on this information, a structural model of the building has to be constructed and then non-linear pushover analysis is used to perform estimation. This process is time-consuming, and technicians are required. It usually takes about one months to estimate the aseismic ability of a building.

In this study, we adopted the WGP method to construct a relation model, which uses the basic design parameters of school buildings to estimate their aseismic ability. The pre-processing efforts carried out for the proposed prediction model can be divided into two parts. The first part is a data filter for selecting out data sets which are reasonable as typical school buildings. The selection conditions of this data filter is proposed based on judgment with expertise in structural engineering. The second part is the determination of the key properties used in the proposed model. We first eliminate some properties which are non-structural and low importance, and synthesize some properties with similarity. We further classifies school building records into subsets based on similarities in property values, and chooses one subset with major population as the data set for further studying. Based on this subset, we try to do further reduction and finally determine 14 properties which are optimal to represent the seismic characteristics of individual school buildings. In comparison with hundreds of properties in the original data, a very high reduction ratio is reached.

Same as GP, a key characteristic of WGP is the resulting model is in the form of mathematic equation, which is very easy and convenient to be applied in engineering practice, and thus is more practical than other soft computing methods. This study also applies GP on the same set of school building data for predicting the building’s collapse ground acceleration. Compared to the prediction results by using GP, the accuracy of WGP model is much better. This case shows WGP can handle problems with more complexities than GP. Regarding the application of WGP method, this study performs 5000 iterations of revolutions for test models with operation tree from two to five levels respectively, and determines the optimized values of model parameters, such as the crossover rate and mutation rate, for this application case through repeated tests. The RMSE of the resulting model achieves less than 0.04. This accuracy in prediction is comparable with the model using Artificial Neural Networks (ANNs). In addition, the graph compared the actual values with predict values shows the distribution trend of the data points are consistent with the expected direction. The result verification indicates the proposed WGP-based model successfully establishes the relation between the 14 input properties and the output property, building’s collapse ground acceleration. In practice, the proposed model can be applied directly and efficiently for preliminary assessment of school buildings for seismic capacities.

In addition to directly inferring the aseismic capacity of school buildings from the design parameters, the current study revised the aseismic capacity index formula of school buildings designed by NCREE using GPS. The formula was to provide greater accuracy with a smaller deviation. This model can help decision makers with issues that are related to the aseismic capacity of school buildings, as well as estimating the disaster loss, during a disaster in a timely manner. These applications require that the estimation of the aseismic capacity of many school buildings be generated rapidly, and this would be impossible if traditional non-linear structural analysis were applied.

\section{D\_isR 值與校舍設計之關係模型}

「校舍是否需要補強」此一指標其數值之基礎即為詳細評估的耐震能力指標 $CDR$ 值,$CDR$ 值超過 1 表示其耐震能力尚符合安全規範,反之,則是有安全疑慮,需要進一步補強或是拆除。而 $D\_isR$ 則是標記各校設耐震能力是否足夠的二元指標,也是教育部的校舍耐震能力補強計畫中,前期篩選工作的最主要目標。找到這個指標與校舍設計、現況等參數的關係模型,與 $Is$ 值和 $CDR$ 值關係模型一樣,此一關係模型對於校舍耐震能力補強計畫中,初期的篩選工作可以有很大的助益,也可以輔助決策者編定預算、快速的根據狀況決定不同計畫年度補強的規模等。

\subsection{資料前處理}
\subsection{資料探勘}
\subsection{結果}

