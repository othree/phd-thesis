\renewcommand\thetable{\arabic{chapter}-\arabic{table}}
%\renewcommand\thefigure{\arabic{chapter}-\arabic{figure}} 
\chapter{校舍資訊與補強經費之關係模型}

校舍建築物的補強經費對於全國校舍的補強工作來說是非常重要的一個數值,和校舍的耐震能力並列,因主管機關每個年度都會編列固定的經費,不過事後難以追加,所以會讓每個年度能夠補強的校舍數量有所限制,對於不知何時會再發生的地震威脅,絕對是希望能夠儘快將最危險的校舍找出並做出對應處置,校舍耐震能力補強的經費編列和分配的影響就非常大,不過實際上在編列預算時,校舍都還沒進行詳細評估、也還沒有建議的補強方案,因此只能透過參考資料和初步評估的資料來估算,如果可以在編列預算時,就知道每間校舍的耐震能力、補強可能需要的經費,那主管機關在編列和分配預算的時候就可以非常有效率的來處理。因此本研究還試著找出校舍的基礎設計參數和他的補強經費間的關係模型。

根據教育部公布 99 年 9 月 16 日修正之國中小校舍「補強工程」之經 費支用範圍及參考單價計價方式,「補強工程」費用,共包括「直接補 強工程費」、「合理之間接修復費」、「工程管理費」、「補強設計及監造費」等費用支出。依公告之估價方式將校舍分為「已完成補強設計」 與「尚未完成補強設計」,歸類為「已完成補強設計」之校舍,其補強工程經費依補強設計審查通過或補強設計報告之金額,教育部原則將據 以如數核定補助經費。但其補強工程單價超出 4000 元/m²者,則應經補強設計期末審查(或特別審查)通過並附審查意見表,始能獲教育部如數核定 補助經費;否則將依單價 4000 元/m²核定補助經費。歸類為「尚未完成補強設計」之校舍(凡未附補強設計審查意見表、未附補強設計報告-重點 摘要部分、或佐證資料不齊全者,皆屬尚未完成補強設計):其「補強工程」之參考單價計價方式依其總樓地板面積(A)(即實際面積)所屬級距範 圍之不同,分別計算如:

