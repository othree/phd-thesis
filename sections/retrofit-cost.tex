\renewcommand\thetable{\arabic{chapter}-\arabic{table}}
%\renewcommand\thefigure{\arabic{chapter}-\arabic{figure}} 
\chapter{校舍資訊與補強經費之關係模型}

校舍建築物的補強經費對於全國校舍的補強工作來說是非常重要的一個數值,和校舍的耐震能力並列,因主管機關每個年度都會編列固定的經費,不過事後難以追加,所以會讓每個年度能夠補強的校舍數量有所限制,對於不知何時會再發生的地震威脅,絕對是希望能夠儘快將最危險的校舍找出並做出對應處置,校舍耐震能力補強的經費編列和分配的影響就非常大,不過實際上在編列預算時,校舍都還沒進行詳細評估、也還沒有建議的補強方案,因此只能透過參考資料和初步評估的資料來估算,如果可以在編列預算時,就知道每間校舍的耐震能力、補強可能需要的經費,那主管機關在編列和分配預算的時候就可以非常有效率的來處理。因此本研究還試著找出校舍的基礎設計參數和他的補強經費間的關係模型。

根據教育部公布九十九年九月十六日修正之國中小校舍「補強工程」之經 費支用範圍及參考單價計價方式,「補強工程」費用,共包括「直接補 強工程費」、「合理之間接修復費」、「工程管理費」、「補強設計及監造費」等費用支出。依公告之估價方式將校舍分為「已完成補強設計」 與「尚未完成補強設計」,歸類為「已完成補強設計」之校舍,其補強工程經費依補強設計審查通過或補強設計報告之金額,教育部原則將據 以如數核定補助經費。但其補強工程單價超出 4000 元/m²者,則應經補強設計期末審查(或特別審查)通過並附審查意見表,始能獲教育部如數核定 補助經費;否則將依單價~4000~元/m²核定補助經費。歸類為「尚未完成補強設計」之校舍(凡未附補強設計審查意見表、未附補強設計報告-重點 摘要部分、或佐證資料不齊全者,皆屬尚未完成補強設計):其「補強工程」之參考單價計價方式依其總樓地板面積(A)(即實際面積)所屬級距範 圍之不同,分別計算如表2.1所示。

\section{資料前處理}

教育部為了提升全國高中職及中小學校舍耐震能力,歷年來已經訂定及執行了各種評估與補強計劃。教育部將計畫委託國震中心執行,國震中心與教育部研擬商討後,再訂出執行計畫交由各縣市政府辦理。計劃的訂定需要資料的輔助,計劃的執行 也需要定期的計劃報告以掌握計劃的執行進度與效力。全國校舍總數, 經目前耐震資料庫中校舍普查作業已記錄之校舍數量統計共有~24930~棟, 如果考慮尚未執行校舍普查作業之校舍,其實際校舍總數將更多。面對 如此大量之校舍,進行總體或縣市為單位之經費估算,需耗費大量時間與人力進行估算與統計。且依照目前「高中職及國中小校舍結構耐震能 力詳細評估作業規範」規定,主管機關須等到詳細評估之承攬廠商提送期末報告書,才能得知該校舍較精確之可能補強費用。因此本研究提出 此子模型來輔助政府單位或主管機關以更短時間來預估大量校舍或於單棟校舍補強作業中之詳細評估初期即能預知可能需要之工程經費,使政 府對於經費之掌握與控管能更加輕鬆與快速。

\subsection{資料篩選}

本研究之子預測模型(工程經費)其構想為使用校舍評估階段初期之已調查資料來針對評估階段後期結果做預估。校舍耐震之補強工程經費須 等到工程完成且承攬廠商提送竣工報告才能得知最後實際施工所花經費, 因研究目標為達成前期預測之目的,所以需挑選補強施工以前之補強階段資料。觀校舍耐震評估與補強之流程如圖 1.1,可知進行補強施工之前尚須進行校舍普查、初步評估、詳細評估與補強設計,經了解各階段評估作業之內容與用途,校舍普查執行之人員為各大專院校之土木系學生 協助執行,且其所記錄之資料也較其他各階段為少,補強設計為補強施工之上一階段,根據「高中職及國中小校舍結構耐震能力補強設計作業 規範」規定,補強設計成果報告書中須編製工程預算書,因此即可於補強設計階段得知該補強工程之所需經費。為達研究之預測目的,本研究將採用初步評估與詳細評估(不含耐震補強方案)結果做為本探勘模型之選擇資料,並以典型校舍、鋼筋混凝土建築、單邊走廊且廊外無柱、無地下室、補強工程經費介於~100~至~1000~萬之間、初步評估與詳細評估均顯示為已送出之資料做為此模型資料選擇之條件篩選。

\subsection{屬性篩選}

本模型採用初步評估與詳細評估(不含耐震補強方案)結果做為本探勘模型之選擇資料,並以典型校舍、鋼筋混凝土建築、單邊走廊且廊外無 柱、無地下室、補強工程經費介於~100~至~1000~萬之間、初步評估與詳細評估均顯示為已送出之資料做為此模型資料選擇之條件篩選。模型經資 料選擇與前處理之後建立資料集,下一步即是進行模型屬性之選擇,建
立屬性集,本模型挑選之屬性集如圖 4.1。模型之屬性分為輸入變數與輸出變數,輸出變數指的就是我們要預測的變數,在本模型即是補強工程之總經費,輸入變數則是我們用來預測輸出變數之所用變數。表 4.1 為本 預測模型屬性集內之屬性介紹與選擇說明。

\begin{multicols}{2}
\begin{itemize}
\item 一樓樓地面積
\item 柱等效強度
\item 軟弱層顯著性
\item 裂縫鏽蝕滲水等程度
\item 短柱嚴重性
\item 調整因子
\item 二樓樓地板面積深
\item 二樓樓地板面積長
\item 樓層數
\item 二樓以上樓地板面積
\item 一樓教室柱根數
\item 一樓走廊外柱斷面積和
\item 非結構牆
\item X正向樑破壞
\item X正向柱破壞
\item X正向窗台柱破壞
\item X正向磚牆破壞
\item X正向性能點之屋頂最大位移
\item X正向性能點之等效基本週期
\item X正向性能點之基底剪力
\item 校舍耐震容量需求比
\item[]
\end{itemize}
\end{multicols}

\begin{description}
  \item[一樓樓地面積]
  依據教育部現行之補強工程計價公式如表~2.1~,其公式以樓地板面積為主要參數,本研究所預測之目 標為實際施工之工程總經費,補強工程計價公式為經費預算之計算方式,估算出來之經費預算往往與實際施工所花經費不太一樣,預算可能高於實際施 工也可能低於實際施工,但因為目標都是估算工程經費,因此本研究將此屬性拿來作為本預測模型之輸入屬性。
  \item[柱等效強度]
  柱之等效強度 TAc 其計算公式為

  \begin{equation}(4+1.8NF)ClaAc+(2.4+1.08NF)CorAc+2.6*InsAc\end{equation} 

  其中~$NF$~為樓層數,~$ClaAc$~為一樓教室柱總斷面積,~$CorAc$~為一樓走廊外柱總斷面積,~$InsAc$~為一樓隔間柱總斷面積,其公式為根據三類 柱之單位面積極限剪力牆度計算公式[30]之加總,本研究將此屬性加入探勘模型之最初原因為該公式內有教室柱、走廊外柱與隔間柱之一樓總斷面積,並且因補強施工其所補強之面積愈大其所花錢就愈 多。本研究經建立多組模型與各種屬性集之測試均發現該屬性對模型之預測結果有一定影響。
  \item[軟弱層顯著性]
  [12] (國震中心,NCREE-03-049)若結構物之一樓因為使用性等考量,而使得二樓以上~RC~牆或磚牆於一樓中斷,致使一樓之極限層剪力強度與勁度降低, 將造成地震力作用時變形集中,以致於韌性用盡,建築物就發生軟弱層破壞。故本表格依據牆體中斷的程度折減其對應之耐震能力,若~2/3~以上牆體中 斷,則耐震能力折減為~0.8~倍;若~1/3~至~2/3~之牆體中斷,則耐震能力折減為~0.9~倍;若~1/3~以下之牆體中斷,則不折減其耐震能力。依此敘述得知軟弱 層顯著性會影響結構物之耐震能力,結構物之耐震能力愈低,極可能所需補強之工程花費就愈高,因此本研究將此屬性納入本屬性集。
  \item[裂縫鏽蝕滲水等程度]
  [12] (國震中心,NCREE-03-049)鋼筋混凝土構材若具有裂縫,代表混凝土品質不良或強度不足;保護層不足等因素使得鋼筋鏽蝕膨脹,鋼筋鏽蝕將會降低 構材之強度,鋼筋鏽蝕膨脹亦會導致混凝土剝落,並加速鋼筋鏽蝕的程度,這些因素都會影響結構物的耐震安全,故以結構物整體之裂縫鏽蝕滲水等程 度作為調整項目。若稍有裂縫鏽蝕滲水等情形,則耐震能力折減為~0.95;若裂縫鏽蝕滲水等情形較為嚴重,則耐震能力折減為~0.9;若無,則不折減其 耐震能力。根據上述,建物若有裂縫鏽蝕滲水等情形會影響耐震能力,耐震能力愈差就有可能提升所需花費之工程經費,因此本研究將之納為輸入屬性。
  \item[短柱嚴重性]
  [12] (國震中心,NCREE-03-049)一般老舊校舍之柱箍筋間距多為~20cm~至~30cm~左右,其剪力強度不高,且老舊校舍於設計時假設為純梁柱系統,並沒有考 慮教室窗台及樓梯廁所等牆壁開氣窗所造成之短柱效應,然而這種短柱效應將會使得剪力容量不足之柱於地震時發生非預期之剪力破壞,導致結構韌性 不足,若該校舍有過多之柱受到短柱效應之影響, 將易造成校舍瞬間倒塌。故若校舍因窗台或氣窗造 成短柱現象之柱根數達到全部柱根數之~50\%~以上, 則耐震能力折減為~0.9~倍,若不足~50\%~則不予折減其耐震能力。值得注意的是,短柱嚴重性具有方向性,故評估時只需考慮評估方向之短柱比率是否超 過一半即可,另一方向開窗等因素造成之短柱效應不需考慮。根據上述得知短柱現象為影響校舍耐震能力之原因之一,因此將之納入探勘模型之中。
  \item[調整因子]
  典型校舍初步評估表之六項調整因子調查項目(平面及立面對稱性、軟弱層顯著性、裂縫鏽蝕滲水等程度、變形程度、平面耐震性、短柱嚴重性),定義~$q_1$~至~$q_6$~分別代表六項調整因子,並定義一整體調整因子~$Q$~為上述六項調整因子之乘積,代表調整因子對於耐震能力的折減或增加。因其功能主要為對 初步評估耐震能力產生增減,所以本研究將其納入屬性集內,並經測試後該屬性對預測之正確性有提升。
  \item[二樓樓地板面積深]
  依據教育部現行之補強工程計價公式如表~2.1~,其公式以樓地板面積為主要參數,又面積之算法為長乘以寬(深),因此亦可將它視為影響校舍補強工程 經費的因素之一。本研究所預測之目標為實際施工之工程總經費,與計價公式所估經費有直接關聯性,因此本研究將此屬性拿來作為本預測模型之輸入屬性。
  \item[二樓樓地板面積長]
  依據教育部現行之補強工程計價公式如表~2.1~,其公式以樓地板面積為主要參數,又面積之算法為長乘以寬(深),因此亦可將它視為影響校舍補強工程 經費的因素之一。本研究所預測之目標為實際施工之工程總經費,與計價公式所估經費有直接關聯性,因此本研究將此屬性拿來作為本預測模型之輸入屬性。
  \item[樓層數]
  樓層數愈高其總樓地板面積愈多,其總樓地板面積愈多,依補強工程計價公式如表~2.1~。其工程經費預算就可能愈多,因此樓層數亦可視為影響工程經 費之影響原因之一。
  \item[二樓以上樓地板面積]
  依據教育部現行之補強工程計價公式如表~2.1~,其公式以樓地板面積為主要參數,本研究所預測之目標為實際施工之工程總經費,補強工程計價公式為 經費預算之計算方式,估算出來之經費預算往往與實際施工所花經費不太一樣,預算可能高於實際施工也可能低於實際施工,但因為目標都是估算工程 經費,因此本研究將此屬性拿來作為本預測模型之輸入屬性。
  \item[一樓教室柱根數]
  教室柱之根數間接可之校舍樓地板面積之大小或規模,因此本研究將之視為影響工程經費之因素之一,經將此屬性納入探勘模型中測試後,也確實發 現此屬性對預測之正確率有一定之提升。
  \item[一樓走廊外柱斷面積和]
  走廊外柱斷面積和愈大其代表意義可能有二:其一,斷面積愈大可能代表樓地板面積愈大,樓地板面積愈大其施工所需經費即可能愈大。其二,如果 校舍樓地板面積大小一樣,但其廊外柱斷面積和比較大,代表其走廊外柱可能斷面積比較大,如果要進行擴柱補強工法,其工程所花經費就會比較高。 綜合以上原因,本研究即嘗試將此屬性納入探勘模型中,並且經測試後效果不錯因此保留此屬性。
  \item[非結構牆]
  此屬性為記錄該校舍,有無非結構牆,並於分析時所建立模型是否有將該非結構牆模擬成其他等值桿件進入結構模型內。因有無非結構牆可能會影響校 舍之耐震能力,因此將之加入輸入變數。
  \item[X~正向樑破壞]
  此屬性為詳細評估資料表中之記錄欄位,記錄詳評後校舍之~X~正向,樑受何種破壞模式,欄內資料分為無破壞、剪力破壞、撓剪破壞、撓曲破壞四種狀 態,不同之破壞模式其需補強所花經費也可能不一樣,因此本研究將之列為本模型之輸入屬性。
  \item[X~正向柱破壞]
  此屬性為詳細評估資料表中之記錄欄位,記錄詳評後校舍之~X~正向,柱受何種破壞模式,欄內資料分為無破壞、剪力破壞、撓剪破壞、撓曲破壞四種狀 態,不同之破壞模式其需補強所花經費也可能不一樣,因此本研究將之列為本模型之輸入屬性。
  \item[X~正向窗台柱破壞]
  此屬性為詳細評估資料表中之記錄欄位,記錄詳評後校舍之~X~正向,窗台柱受何種破壞模式,欄內資料分為無破壞、剪力破壞、撓剪破壞、撓曲破壞四 種狀態,不同之破壞模式其需補強所花經費也可能不一樣,因此本研究將之列為本模型之輸入屬性。
  \item[X~正向磚牆破壞]
  此屬性為詳細評估資料表中之記錄欄位,記錄詳評後校舍之~X~正向,磚牆受何種破壞模式,欄內資料分為無破壞、剪力破壞、撓剪破壞、撓曲破壞四種 狀態,不同之破壞模式其需補強所花經費也可能不一樣,因此本研究將之列為本模型之輸入屬性。
  \item[X~正向性能點之屋頂最大位移]
  [14]進行詳細評估時,若採用 PushOver 分析,則該欄位則須依~X~及~Y~方向分別輸入屋頂最大位移值。本研究僅挑選採用 PushOver 分析之校舍進行建模, 因屋頂之最大位移為影響校舍耐震能力的因素之一,因此也將該屬性列入。
  \item[X~正向性能點之等效基本週期]
  [31]基本週期為結構物等效單自由度系統的動力參數(性能目標地表加速度),此單自由度系統在性能目標地表加速度的設計地震作用下,其動力反應將 是已設定的性能需求,因此基本週期亦是影響耐震能力評估結果的因素之一。
  \item[X~正向性能點之基底剪力]
  基底剪力為側推分析中計算容量曲線之重要參數之一,校舍之耐震能力可能影響補強之工程經費,因此本研究嘗試將此屬性放入本模型中進行測試,其 結果對模型預測之正確率有提升,因此將它保留至屬性集之中。
  \item[校舍耐震容量需求比]
  [15]校舍建築物進行耐震初步評估後,得到一耐震指標分數,此耐震指標可作為是否需進入下一階段評估工作的參數,同樣地,經由專業人員完成詳細 評估後,得到建築物長短向最大可抵抗之地表加速度,除以該校舍~475~年回歸期設計地表加速度,稱之耐震容量需求比(Capacity Demand Ratio),此數值 若小於一時,則判定該棟建築須進行補強;反之,該棟建築耐震能力暫無疑慮,但仍須定期檢視。根據上述得知~CDR~可視為該校舍之耐震能力與正常 安全值之距離,如果數值小於~1~且離~1~愈遠,代表需替該校舍提升的耐震能力愈多,所需花的經費就愈多,因此亦可視為影響補強工經費之因子。
\end{description}

\section{資料探勘}

\section{結果}

