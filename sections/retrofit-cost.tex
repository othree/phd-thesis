\renewcommand\thetable{\arabic{chapter}-\arabic{table}}
%\renewcommand\thefigure{\arabic{chapter}-\arabic{figure}} 
\chapter{校舍資訊與補強經費之關係模型}

校舍建築物的補強經費對於全國校舍的補強工作來說是非常重要的一個數值,和校舍的耐震能力並列,因主管機關每個年度都會編列固定的經費,不過事後難以追加,所以會讓每個年度能夠補強的校舍數量有所限制,對於不知何時會再發生的地震威脅,絕對是希望能夠儘快將最危險的校舍找出並做出對應處置,校舍耐震能力補強的經費編列和分配的影響就非常大,不過實際上在編列預算時,校舍都還沒進行詳細評估、也還沒有建議的補強方案,因此只能透過參考資料和初步評估的資料來估算,如果可以在編列預算時,就知道每間校舍的耐震能力、補強可能需要的經費,那主管機關在編列和分配預算的時候就可以非常有效率的來處理。因此本研究還試著找出校舍的基礎設計參數和他的補強經費間的關係模型。

根據教育部公布 99 年 9 月 16 日修正之國中小校舍「補強工程」之經 費支用範圍及參考單價計價方式,「補強工程」費用,共包括「直接補 強工程費」、「合理之間接修復費」、「工程管理費」、「補強設計及監造費」等費用支出。依公告之估價方式將校舍分為「已完成補強設計」 與「尚未完成補強設計」,歸類為「已完成補強設計」之校舍,其補強工程經費依補強設計審查通過或補強設計報告之金額,教育部原則將據 以如數核定補助經費。但其補強工程單價超出 4000 元/m²者,則應經補強設計期末審查(或特別審查)通過並附審查意見表,始能獲教育部如數核定 補助經費;否則將依單價 4000 元/m²核定補助經費。歸類為「尚未完成補強設計」之校舍(凡未附補強設計審查意見表、未附補強設計報告-重點 摘要部分、或佐證資料不齊全者,皆屬尚未完成補強設計):其「補強工程」之參考單價計價方式依其總樓地板面積(A)(即實際面積)所屬級距範 圍之不同,分別計算如表2.1所示。

\section{資料前處理}

教育部為了提升全國高中職及中小學校舍耐震能力,歷年來已經訂定及執行了各種評估與補強計劃。教育部將計畫委託國震中心執行,國震中心與教育部研擬商討後,再訂出執行計畫交由各縣市政府辦理。計劃的訂定需要資料的輔助,計劃的執行 也需要定期的計劃報告以掌握計劃的執行進度與效力。全國校舍總數, 經目前耐震資料庫中校舍普查作業已記錄之校舍數量統計共有 24930 棟, 如果考慮尚未執行校舍普查作業之校舍,其實際校舍總數將更多。面對 如此大量之校舍,進行總體或縣市為單位之經費估算,需耗費大量時間與人力進行估算與統計。且依照目前「高中職及國中小校舍結構耐震能 力詳細評估作業規範」規定,主管機關須等到詳細評估之承攬廠商提送期末報告書,才能得知該校舍較精確之可能補強費用。因此本研究提出 此子模型來輔助政府單位或主管機關以更短時間來預估大量校舍或於單棟校舍補強作業中之詳細評估初期即能預知可能需要之工程經費,使政 府對於經費之掌握與控管能更加輕鬆與快速。

\subsection{資料篩選}

本研究之子預測模型(工程經費)其構想為使用校舍評估階段初期之已調查資料來針對評估階段後期結果做預估。校舍耐震之補強工程經費須 等到工程完成且承攬廠商提送竣工報告才能得知最後實際施工所花經費, 因研究目標為達成前期預測之目的,所以需挑選補強施工以前之補強階段資料。觀校舍耐震評估與補強之流程如圖 1.1,可知進行補強施工之前尚須進行校舍普查、初步評估、詳細評估與補強設計,經了解各階段評估作業之內容與用途,校舍普查執行之人員為各大專院校之土木系學生 協助執行,且其所記錄之資料也較其他各階段為少,補強設計為補強施工之上一階段,根據「高中職及國中小校舍結構耐震能力補強設計作業 規範」規定,補強設計成果報告書中須編製工程預算書,因此即可於補強設計階段得知該補強工程之所需經費。為達研究之預測目的,本研究將採用初步評估與詳細評估(不含耐震補強方案)結果做為本探勘模型之選擇資料,並以典型校舍、鋼筋混凝土建築、單邊走廊且廊外無柱、無地下室、補強工程經費介於 100 至 1000 萬之間、初步評估與詳細評估均顯示為已送出之資料做為此模型資料選擇之條件篩選。

\subsection{屬性篩選}

本模型採用初步評估與詳細評估(不含耐震補強方案)結果做為本探勘模型之選擇資料,並以典型校舍、鋼筋混凝土建築、單邊走廊且廊外無 柱、無地下室、補強工程經費介於 100 至 1000 萬之間、初步評估與詳細評估均顯示為已送出之資料做為此模型資料選擇之條件篩選。模型經資 料選擇與前處理之後建立資料集,下一步即是進行模型屬性之選擇,建
立屬性集,本模型挑選之屬性集如圖 4.1。模型之屬性分為輸入變數與輸 54
出變數,輸出變數指的就是我們要預測的變數,在本模型即是補強工程之總經費,輸入變數則是我們用來預測輸出變數之所用變數。表 4.1 為本 預測模型屬性集內之屬性介紹與選擇說明。

\begin{multicols}{2}
\begin{itemize}
\item 一樓樓地面積
\item 柱等效強度
\item 軟弱層顯著性
\item 裂縫鏽蝕滲水等程度
\item 短柱嚴重性
\item 調整因子
\item 二樓樓地板面積深
\item 二樓樓地板面積長
\item 樓層數
\item 二樓以上樓地板面積
\item 一樓教室柱根數
\item 一樓走廊外柱斷面積和
\item 非結構牆
\item X正向樑破壞
\item X正向柱破壞
\item X正向窗台柱破壞
\item X正向磚牆破壞
\item X正向性能點之屋頂最大位移
\item X正向性能點之等效基本週期
\item X正向性能點之基底剪力
\item 校舍耐震容量需求比
\item[]
\end{itemize}
\end{multicols}

\begin{description}
  \item[一樓樓地面積]
  依據教育部現行之補強工程計價公式如表 2.1,其公式以樓地板面積為主要參數,本研究所預測之目 標為實際施工之工程總經費,補強工程計價公式為經費預算之計算方式,估算出來之經費預算往往與實際施工所花經費不太一樣,預算可能高於實際施 工也可能低於實際施工,但因為目標都是估算工程經費,因此本研究將此屬性拿來作為本預測模型之輸入屬性。
  \item[柱等效強度]
  \item[軟弱層顯著性]
  \item[裂縫鏽蝕滲水等程度]
  \item[短柱嚴重性]
  \item[調整因子]
  \item[二樓樓地板面積深]
  \item[二樓樓地板面積長]
  \item[樓層數]
  \item[二樓以上樓地板面積]
  \item[一樓教室柱根數]
  \item[一樓走廊外柱斷面積和]
  \item[非結構牆]
  \item[X正向樑破壞]
  \item[X正向柱破壞]
  \item[X正向窗台柱破壞]
  \item[X正向磚牆破壞]
  \item[X正向性能點之屋頂最大位移]
  \item[X正向性能點之等效基本週期]
  \item[X正向性能點之基底剪力]
  \item[校舍耐震容量需求比]
\end{description}

\section{資料探勘}

\section{結果}

