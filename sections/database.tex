\renewcommand\thetable{\arabic{chapter}-\arabic{table}}
%\renewcommand\thefigure{\arabic{chapter}-\arabic{figure}}
\renewcommand{\theequation}{\arabic{chapter}-\arabic{equation}}
\chapter{校舍耐震資料庫}

學校是人才培育的場所,也是緊急災難時,居民避難的主要地方,但台灣地區學校建築在每次地震時之損壞卻非常嚴重,尤其是老舊校舍,因興建年代久遠,其設計時所依據規範較為老舊,耐震能力可能遠低於現今結構耐震安全上之要求,故教育部已委託國家地震工程研究中心進行全國學校校舍之耐震能力評估與補強研究,此研究著眼於校舍建築耐震能力評估補強機制及施行細節的建立,以對全國學校校舍耐震能力作一全面性普查,以篩選出耐震有疑慮之校舍,並儘速透過補強或拆 除新建的手段來提昇校舍的耐震能力。

\section{建置目的} 

結構耐震評估的方法可分為兩類,第一類為初步評估(preliminary evaluation),其通常的方法是基於結構物之設計及現況填寫初評表格,所填寫資料再依評估方程式計算出結構物耐震能力之評分等級或指數,此類方法主要的目的是對大量結構物之耐震能力作排序與篩選;另一類為詳細評估(detailed evaluation),此類方法為對結構物進行詳細的結構耐震分析,通常是使用結構分析程式以電腦數值計算的方式,準確詳細的檢驗評估出結構物的耐震能力,此二類耐震評估的方式通常相輔相成,結合成為標準的結構耐震能力評估程序,其即是先對所有需評估之結構物先以初步評估作耐震能力之評分與排序,以篩選出其中耐震能力有疑慮者,再對其以詳細評估的方式做詳細的檢驗。

由於中小學校舍數量龐大,若直接大量投入人力物力,可能造成大量之資源浪費,也無法快速的鎖定耐震能力不足之校舍建築,故針對有效達成此一校舍耐震評估標準需求以及基於上述標準耐震評估程序之精神,國家地震工程研究中心提出之解決程序為:經由學校總務人員之簡易調查及工程專業人員之初步評估,有效的將校舍結構之耐震能力排序,以縮小問題之規模,對於耐震堪慮之校舍,依嚴重程度,由工程之專業人員,進行結構耐震之詳細評估,倘尚符合補強之經濟效益,即進一步作耐震補強之設計,若不符合補強之經濟效益,則將之列為拆除重建。

上述程序中之簡易調查與初步評估均屬填寫評估表方式之初步篩選階段,此一調查評估作業,將產生大量與校舍結構耐震相關之資料,包含GPS座標、樓地板面積、柱量、牆量、興建年代、用途等,而後續之詳細耐震評估與補強工作,亦會產生大量分析模型,補強設計圖說等多 媒體資料。故本研究的目標之一,即是針對以上校舍耐震評估作業中所欲收集及可能產生之資料,規劃建置校舍耐震資料庫,以對校舍耐震評 估作業所產生大量的珍貴資料作有系統的維護、儲存與管理,以供研究團隊後續研擬詳細評估與補強作業之方法與策略,以及其他防災計畫與 建物耐震相關研究參考使用,亦可為填表式的評估作業提供一個網路上傳式的作業平台,輔助資料收集與彙整之工作,以提昇作業之方便性, 正確性及效率,並節省人力物力成本。

\subsection{簡易調查} 
