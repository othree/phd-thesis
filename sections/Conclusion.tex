\renewcommand\thetable{\arabic{chapter}-\arabic{table}}
%\renewcommand\thefigure{\arabic{chapter}-\arabic{figure}} 
\chapter{結論與討論}
\label{cha:conclusions}


\section{結論}

Conclusion


\section{未來展望} 



資料探勘可分為四種知識型態,包括了迴歸、分類、分群、關聯四種,本研究初期依據此四種型態分析設計了數個不同面向的探勘目標,其中迴歸分析中,主要的知識為重要屬性的關係模型,例如本研究中的耐震能力、補強經費與校舍基本設計參數間之關係模型即為此一類之資料探勘。而分類分析中,目前最主要的分析是校舍是否安全、是否需要補強的分類探勘,即為本研究中所建立之校舍基礎設計參數與是否需要補強間之關係模型。分群類型之知識主要在於校舍結構之不同群集,在本研究中,建立校舍設計參數與其~$Is$~值之關係模型時,便有使用到此一知識來增進關係模型的品質,後續研究還可以針對這些校舍群集,進一步分析其不同群集之特性,至於關聯形式的知識,本研究目前尚未有可用的知識產出,因此後續的研究方向也包含此一類型知識的探勘與挖掘。

\begin{figure}[hbtp]
  \begin{center}
    \includegraphics[width=1.0\textwidth]{figures/big-picture.pdf}
    \caption{知識挖掘規劃} 
    \label{fig:bigpicture}
  \end{center}
\end{figure}

另外一個發展方向,則是在 CRISP-DM 流程當中的最後一個步驟,將探勘得到的知識實際回饋到學校校舍及相關設備補強整建計畫上,由於目前探勘得到的知識都還是以數學模型的形式存在,非專業人士難以應用,因此如果可以將這些數學模型轉化成決策支援系統,則可以讓主管機關能夠簡單的得到這些模型的輔助,在校舍長期持續的耐震能力監控上,能夠發揮探勘所得知識的效力。


