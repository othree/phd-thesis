%\renewcommand\thefigure{\arabic{chapter}-\arabic{figure}}
%\bibliographystyle{unsrt} 
\chapter{緒論}
\label{cha:intro} 

\section{動機與目的}

中小學校舍是使用人數密度極高的建築,且在國家防災的規劃上,許多的學校都是災害發生時,用來收容與暫時安置災民的重要場所,因此其建築物之可靠度相當重要,然而一九九九年九月二十一日發生的南投集集大地震,造成台灣將近半數校舍受損,根據統計,中小學遭到損壞者共計656所,約佔全國學校總數的五分之一,其中南投地區更是有多數的校舍半毀或全毀,所幸地震發生時間為半夜,校舍並未在使用時間,並沒有因為校舍的受損而造成學生的傷亡,但是學校校舍的耐震能力是否足夠的問題也因此浮現。

現在中小學學校校舍的耐震性不足其問題最主要在於有很大比例的校舍,其屋齡已經很大,建造時使用的是過去的建築規範,其耐震能力以現今的建築法規來看就顯得不足,因此教育部國教司在九二一大地震後,便開始了校舍耐震能力補強計畫,並與國家研究院地震工程研究中心(NCREE)合作執行,目的在找出所有耐震能力有疑慮的學校校舍,根據評估結果來補強其安全性,如果有需要,甚至是拆除重建,然而要詳細評估建築物之耐震能力,需要使用如 FEMA-273\cite{building1997nehrp} 規範建議之非線性推垮分析來評估,而非線性推垮分析之正確性仰賴完整且詳細的數值模型,模型的建立則是高度仰賴專業人力的耗時工作,因此非線性推垮分析非常的耗時且非常昂貴。

此一校舍耐震能力補強計畫則是用三階段的篩選機制來快速的篩選出耐震能力有疑慮的校舍,進行詳細的耐震能力評估並進入後續的處理,這個篩選流程的第一階段為全國中小學校校舍之普查與建檔,並調查了非常基本的校舍資訊,第二階段則是初步評估,是委託各個地方的專業人士,如土木、結構技師、建築師等,到各學校透過NCREE所設計之初步評估表格,快速的推算出校舍的耐震能力參考值,藉以判斷該校舍是否有安全疑慮,是否需要進入下一個階段的評估流程。第三個階段是詳細評估,這個階段使用的評估方法就是非線性推垮分析,由專業人士去現地調查,建立出完整詳細的數值模型,並用電腦模擬分析,評估校舍的耐震能力,最後才根據這個數值決定校舍是否需要補強,如果需要補強,則進入到之後的補強流程。

在這個計畫執行的過程中,不同階段的評估都會產生大量的校舍相關資料,例如初步評估會有校舍的基本結構參數:長度、深度、樓層數、柱尺寸、柱數量以及校舍現狀等等資料,詳細評估則會有更詳細的如材料強度、破壞構件、破壞地表加速度等等,後續的補強流程還會產生如補強方法、補強量、補強經費等等資料,數量龐大,因此國家地震工程研究院便建置了一個校舍耐震資料庫,收集此計畫執行間產生的各種校舍資料,目前資料庫收集有全台灣兩萬多棟校舍的設計、評估與工程相關資料,其主要用途雖為輔助校舍耐震能力補強計畫,然而此一大量的資料,應當還可以從中挖掘出難以由人工觀察判讀的隱含知識。

資料探勘(Data mining)此一研究領域的發展是為了因應資料庫系統以及資料倉儲系統的發展、資料量的急遽成長以及越來越複雜的資料性質,因而越來越難從收集的資料中獲取有用的知識的情形。資料探勘的方法包括統計、線上分析處理(OLAP、on-line analytical processing)、情報檢索(information retrieval)、機器學習(machine learning)、模式識別(pattern recognition)等,由前段敘述可以得知,校舍耐震資料庫內的資料量非常多,不只校舍數量旁大,收集的資料屬性也非常多,其中隱含的知識難以直接由人眼觀察取得,如果可以使用資料探勘技術,從其各種分析方法的特性出發,配合各種實務上的需求,應當可以從此資料庫中找出部分隱含的校舍建物知識。


Data-mining methods were developed in response to the development of database systems and data warehousing systems, the rapid growth in data volume, and the increasingly complexity of data. Hence, extracting useful knowledge from a collection of statistics has become increasingly difficult. Modern of data-mining techniques include statistics, on-line analytical processing (OLAP), information retrieval, machine learning, and pattern recognition. Prediction models, clustering models, and data patterns can be generated from such methods. As mentioned, assessing the aseismic ability of buildings is time-consuming and must be conducted by professionals. Thus, a model that can predict aseismic ability using data-mining technology based on existing assessment examples would be faster than nonlinear analysis of buildings and have satisfactory accuracy.

As buildings have various forms without a fixed pattern, and their design, layout, and dimensions cannot be classified and quantified, converting building information to fixed and quantifiable attributes is almost impossible. However, few studies have attempted to construct a model for predicting the unique attributes of an entire building, such as its aseismic ability, lifespan, and maintenance costs. This work focuses on typical school buildings that follow certain design rules and have fixed design patterns. The design and geometric configuration of buildings can be expressed by several quantitative characteristics, which are utilized to train and build a model for predicting building features. School buildings have regular patterns, and most have similar designs such as an I-shape with external corridors and partitions. Hence, complex and varied design, dimensions, and other building data can be transformed into typical properties, such that a prediction model can be constructed based on overall school building features. Such a prediction model can be constructed using heuristic computing technology. Taiwan’s National Center for Research on Earthquake Engineering (NCREE) constructed a database of aseismic information for school buildings and collected data for 20,000 school buildings in Taiwan, including data for structure, design (e.g., beam column design), quantity of structure elements, design pattern, number of floors, and distribution of classrooms, in addition to the floor layout, usage of building, and number of occupants. These data are for specific structural features of school buildings. The aseismic ability of roughly 1000 school buildings has been assessed by professionals. Based on this database of school buildings and aseismic ability data obtained through nonlinear analysis, this work builds a novel prediction model for determining the aseismic ability of school buildings using data-mining technology.

在各式建築物結構中,校舍為結構形式上較具有規則性之一類,大多數的校舍建築之設計形式相近,為一字型、外有走廊、隔間形式類似,因此可以將建築物原本複雜多樣化的設計、尺寸等資訊轉為數個代表屬性,此一特性讓校舍建築物整體特性預測模型的建置變成可能,可以使用各種軟式運算(heuristic computing)技術來建置預測模型。台灣的國家地震工程研究中心(NCREE, National Center for Research on Earthquake Engineering)已建立有一校舍耐震資料庫,此一資料庫收集有全台灣約兩萬棟校舍的各種資料,除了學校和校舍的基本資料如學校位置,校舍用途、使用人數等,還收集了各校舍的結構、設計資訊,如梁柱尺寸(beam column design)、數量、設計形式(design pattern)、樓層數、教室分布等,這些資料屬性皆為針對校舍特定建築形式之特性所定。

本研究之研究目的即為基於校舍的建築形式以及已經收集大量校舍資料的校舍耐震資料庫,利用各種資料探勘方法,針對校舍耐震資料庫挖掘隱藏於其中的隱含知識。

\section{研究方法}

本研究之研究方法可以分為三個階段,第一個階段為分析規劃階段,此一階段的主要目標為假設各種可能的隱含知識,並且定義出不同隱含知識的探勘方式,Fayyad~\cite{fayyad1996data} 依照不同資料探勘技術之特性,分出迴歸、分類、分群、關聯四大類,本研究的第一階段即根據此四種知識形式,以及校舍耐震補強計畫的執行流程與需求,假設並定出各種可能透過資料探勘技術取得的校舍耐震資料庫隱含的知識以及探勘方法,其中,迴歸形式的可能取得知識包括了校舍耐震能力預測、校舍破壞模式預測、校捨補強經費預測等,分類形式的可能知識包括了校舍是否需要補強的預測,分群形式的知識則是校舍的類型歸類條件,關聯式法則形式的可能知識則是校舍設計參數與其現狀的關連性。

第二階段則是根據假設的各種隱含知識和資料探勘規劃,實際進行資料探勘的分析和測試,最後的第三階段則是探勘結果的驗證和隱含知識的整理,基於此一流程,本研究最後得到了三個隱含於校舍耐震資料庫中的知識,分別為:

\begin{itemize}
\item 耐震能力與校舍設計關係模型
\item 校舍設計、現況與破壞構件之關係模型
\item 校舍設計、現況與補強經費之關係模型
\end{itemize}

校舍耐震能力預測模型為本研究最主要的資料探勘目標,因為校舍耐震能力補強計畫當中,最重要的資訊就是校舍的耐震能力,傳統上,如果要取得可靠的校舍耐震能力,需要由專業的技師來評估,其過程需要先到現場調查,根據調查的資續建立完整的結構數值模型,並使用非線性的分析軟體分析,其過程耗時且所費不貲,因此現在校舍耐震能力補強計畫是以分階段篩選的機制,先讓所有校舍進行一個較為簡單的初步評估,再根據初步評估的結果來決定哪些校舍的耐震能力可能比較不足夠,需要詳細的非線性分析,才真的對這些校舍進行詳細的非線性分析與耐震能力評估,然而這種方法有個缺點是其初步的評估方法無法完全反映出校舍的耐震能力,可能有校舍已經因為年代久遠造成耐震能力低落,然而卻無法在初步評估的結果中真實的反映出來,因此,如果有一個方法可以快速的得到更為可靠的評估數據,甚至可以當作詳細評估的參考,可以大大的加速校舍耐震能力補強計畫的進行。

除了數值化的校舍耐震能力,本研究還建立一個預測模型,可以對校舍受到地震力時,優先破壞的構件進行預測,這個資訊可以幫助對校舍進行耐震能力評估的專業技師對目標的校舍弱點先有一些初步了解,不但可以協助詳細評估的進行,對於校舍補強設計的方式也有一定程度的幫助。

最後,由於校舍補強所需的經費龐大,因此校舍耐震能力補強計畫不可能在短期內就把所有耐震能力不足的校舍都完成補強,實務上會需要估算各個校舍補強所需的經費,排定預算,然後才知道不同預算年度能夠完成多少的校舍補強作業,因此校舍的補強經費在校舍補強計畫的決策中,是一個非常重要的數字,傳統的經費預估方法是由過往的經驗、數據和所欲補強校舍的規模作為依據,經由一些推估和統計所計算出來的,如果能夠建立一個預測模型,經由校舍的基本資料就可以得到準確的補強經費預測值,那便可以大大的加速校舍耐震能力補強計畫決策者的決策速度,也可以讓計畫執行人員能更快的了解補強作業的規模。

\section{論文架構}

本論文共分為八章,各章內容分別介紹如下:

\begin{enumerate}
\item 緒論:說明本研究之動機及目的。
\item 相關研究:回顧與本研究相關之文獻。
\item 校舍耐震資料庫:介紹校舍耐震資料庫之架構與其所收集資之資料。
\item 資料探勘:介紹資料探勘技術與本研究使用到的各種演算法。
\item 耐震能力與校舍設計關係模型:詳述本研究第一個使用資料探勘方法找到的隱含知識,耐震能力與校舍設計參數間的關係模型。
\item 校舍設計、現況與破壞構件之關係模型:詳述使用資料探勘技術找出校舍現況、設計參屬等屬性與其遇到地震力時,可能先受力破壞構件之關係模型之方法與過程
\item 校舍設計、現況與補強經費之關係模型:詳述使用資料探勘技術找出校舍現況、設計參數等屬性與其可能需要之補強經費間關係模型之方法與過程
\item 結論:結果探討與未來展望。
\end{enumerate}



網路發展興盛至今,小至個人,大至政府單位與各機關組織,都相當仰賴網路的使用,但許多人仍然對資安危機意識較低,針對資訊安全產品的投資也相對較少,加上對於資訊安全軟體工具缺乏有系統的整理,以致於未能有效運用。為此,本手冊蒐集整理相關開放源碼(Open Source)的資訊安全軟體工具,並透過專業人員實際操作演練,加以彙整並集結成冊,希冀透過本手冊的幫助,不僅能給予初學者對於資安工具軟體初步認識,也讓資訊從業人員在資訊安全工具上能有更多的選擇與應用。

資安開放源碼軟體的發展,往往會公開其發展技術及運用的原理,配合程式碼的開放,使得開放源碼軟體具有相當大的彈性,並根據個人使用情況所需,進行軟體的編修與整合,以求適應各種作業環境所需。使用開放源碼軟體所需負擔的金錢成本,遠低於商業付費軟體,可降低企業組織對資訊科技產品的部分支出,不需要過度仰賴軟體製造商的技術支援與更新,也能減少相對應的軟體開發時程。由於目前多數的資安開放源碼軟體的開發多為國外組織,因此較缺乏中文化介面,且部分軟體工具的使用,需要具備相當程度的專業知識,並非人人皆可輕易上手。本手冊擬透過中文化的工具介紹,減緩國內使用者入門的負擔。

由於現今網路環境日益複雜,遭受網路攻擊的事件層出不窮,網路安全越來越受到各界重視。網路掃描是網路安全的根本,也是攻擊者對目標主機進行攻擊的首要步驟,因此,了解網路掃描的攻擊與防禦,將有助於網路管理者提升網域的安全管理。此外,網路流量代表所有網路訊息的傳送,能提供管理者即時了解網路狀況,藉此檢視網路情況正常與否。本手冊將針對以上兩類的開放源碼軟體,逐一介紹其功能、安裝、操作與軟體評比,令讀者對相關的資訊軟體能有所了解,並進一步應用於資訊安全的監測與控管。以下即對網路掃描及流量監控兩大類軟體,進行整理與原理說明。
	
如表~\ref{tab:system}所示。 


\begin{table}[hbtp]
  \begin{center}
    \caption{The relation of aggregation overhead between different techniques}
    \label{tab:system}
    \begin{tabular}{|c|c c c|}
      \hline
       & Space usage & Communication & Query \\
       & of root aggregator & overhead & requirement \\
      \hline
      Traditional warehouse & $n$ & $O(n)$ & $O(n)$ \\
      \hline
      AM-FM sketch technique & $\log a$ & $O(\log n)$ &  $O(a\log n)$ \\
      \hline
      ``prototypical PHI query'' & $\log a$ & $O(\log n)$ & $O(\log n)$ \\
      \hline
      \end{tabular}
  \end{center}
\end{table}
