%\renewcommand\thefigure{\arabic{chapter}-\arabic{figure}}
%\bibliographystyle{unsrt} 
\chapter{緒論}
\label{cha:intro} 

TODO

\begin{itemize}
\item Big Picture 移到前面
\item 失敗案例介紹(分群、關聯)
\item 參數敏感度
\item 子群集比例
\item 結果加強:破壞構件模式結果、 Is 與補強關係 ...
\item CHAID, Apriori 介紹
\item 更詳細的資料探勘流程
\item 研究方法詳細介紹
\item 資料時間點
\item 符號表?
\end{itemize}

\section{動機與目的}

中小學校舍是使用人數密度極高的建築,且在國家防災的規劃上,許多的學校都是災害發生時,用來收容與暫時安置災民的重要場所,因此其建築物之可靠度相當重要,應比一般建築物有更高的要求,然而一九九九年九月二十一日發生的南投集集大地震,造成台灣將近半數校舍受損,讓學校校舍耐震能力不足的隱憂浮現,根據統計,中小學遭到損壞者共計~656~所,約佔全國學校總數的五分之一,其中南投地區更是有多數的校舍半毀或全毀,所幸地震發生時間為半夜,校舍並未在使用時間,並沒有因為校舍的受損而造成學生的傷亡。

現在中小學學校校舍的耐震性不足其問題最主要在於有很大比例的校舍屋齡已經很大,建造時的建築規範對於耐震能力的要求以現今的建築法規來看已經顯的不足,二來老舊校舍也會因為長期的使用以及經過數次天災、地震的影響,而使的其可靠度下降,因此教育部國教司在集集大地震後,就開始並持續的與國家實驗研究院國家地震工程研究中心(國震中心、National Center for Research on Earthquake Engineering, NCREE)合作執行「加速國中小老舊校舍及相關設備補強整建計畫」,目的在找出所有耐震能力有疑慮的學校校舍,請專業人士來評估現有校舍建築的耐震能力,根據評估結果來判斷是否有安全疑慮需要補強,抑或是需要拆除重建,後續更執行了「加速高中職及國中小老舊校舍及相關設備補強整建計畫」,進一步拓展評估補強的範圍到更多的學校單位,這些相關的計畫在本研究則統稱為「校舍耐震能力補強計畫」。

現有結構非線性分析模擬技術之發展已經能掌握結構系統於地震力作用下之整體反應與其細部構件之行為,使的非線性分析成為評估結構耐震能力最為詳細可靠的方法,美國~FEMA-273\cite{building1997nehrp}~之規範已建議使用非線性推垮分析來評估結構物於不同程度地震力作用下之破壞形式與可靠度,然而非線性推垮分析非常的耗時且昂貴,因其分析之正確性仰賴完整且詳細的數值模型以及專業人士的操作與結果判讀,模型的建立也同樣也是高度仰賴專業人力的耗時工作,要對全國所有校舍都進行專業的詳細評估實在是有所困難,且耗時長久,在地震威脅不會消失的情況下,要對全國所有的學校校舍都進行非線性推垮分析,實在是緩不濟急。

因此校舍耐震能力補強計畫是用三階段的篩選機制來快速的篩選出耐震能力有疑慮的校舍,針對挑選出來的校舍先詳細的評估其耐震能力並根據結果決定後續的處理,這個篩選流程的第一階段為全國中小學校校舍之普查與建檔,且同時調查校舍的基本資訊,第二階段則是初步評估,是委託各個地方的專業人士,如土木、結構技師、建築師等,到各學校透過國震中心所設計之初步評估表格,快速的推算出校舍的耐震能力參考值,藉以判斷該校舍是否有安全疑慮,如有安全疑慮則需要進入下一個階段的詳細評估流程。第三個階段是詳細評估,這個階段使用的評估方法就是非線性推垮分析,由專業人士去現地調查,建立出完整詳細且能反映現況的校舍數值模型,之後使用非線性推垮分析方法來模擬校舍受到地震力之行為,評估校舍的耐震能力,最後才根據評估結果決定校舍是否需要補強,如果需要補強,則進入到之後的補強流程,而耐震能力嚴重不足的校舍,則會建議拆除重建。

在這個計畫執行的過程中,四個不同階段的工作都會產生大量的校舍相關資料,例如初步評估會有校舍的基本結構參數:長度、深度、樓層數、梁柱之尺寸及數量、樓地板面積以及校舍現狀等等資料,詳細評估則會有更詳細的如材料強度、優先破壞的構件、性能點最小破壞地表加速度等等,後續的補強流程還會產生如補強工法、不同工法的補強量、補強經費等等,資料量龐大,因此國震中心便建置了一個校舍耐震資料庫,統一收集普查、評估到補強完成之間所產生的各種校舍資料,目前這個資料庫收集有全台灣兩萬多棟校舍的設計、評估與工程相關資料,其主要用途雖為輔助校舍耐震能力補強計畫,然而此一大量的資料,應當還有很多隱含於其中、難以由人工觀察判讀得到的隱含知識。

資料探勘(data mining)此一研究領域的發展是為了因應由於資料庫系統以及資料倉儲系統的發展、資料量的急遽成長以及越來越複雜的資料性質,因而越來越難從收集的資料中獲取有用的知識的情形。資料探勘的方法包括統計、人工智慧(artificial intelligence)、機器學習(machine learning)、模式識別(pattern recognition)等,由前段敘述可以得知,校舍耐震資料庫內的資料量龐大,不只校舍數量多,收集的資料屬性也非常多樣,其中有隱含尚未為人知之知識的可能性非常高,但是難以直接由人工觀察取得,因此如果可以使用資料探勘技術,從其各種分析方法的特性出發,配合各種實務上的需求,應當可以從此資料庫中找出隱含的校舍建物知識。

然而資料探勘分析需要每筆資料都能夠用相同的形式,並且用固定數量且有限的資料屬性來描述資料實體的特性,對於建築物資料來說,不同的建築物,結構差異可能很大,沒有一個標準的形式可以只使用有限的資料屬性就描述所有的建築物,因此以往的研究均難以對大量的建築物資料進行資料探勘分析,不過校舍建築中,有很大的比例有相似的結構形式,這些校舍都為一字形,隔間為教室一間接著一間連著,外面有走廊,樓梯間、廁所通常在末端,樓層數不超過五層樓,有些校舍雖然非一字形,較為複雜可能是~L~字形或是ㄇ字形,但是也可拆分為數個一字形形式的校舍,而由於有這些常見的形式,可以把校舍建築的資料屬性特徵化,用少量的資料屬性就可以正確的描述這種常見形式的校舍建築,資料屬性數量不會隨著建築物規模的擴大而增加,這樣的資料形式讓大量校舍建築物的資料探勘分析成為可能,而國震中心所建立的校舍耐震資料庫中收集的校舍資料即為使用這種形式來描述校舍設計參數之資料,因此本研究之研究目的即為基於校舍的建築形式以及已經收集大量校舍資料的校舍耐震資料庫,利用各種資料探勘方法來分析並尋找此一資料庫中,難以人工觀察判讀的隱含知識。

\section{研究方法}

本研究之研究方法可以分為三個階段,第一個階段為分析規劃階段,此一階段的主要目標為假設各種可能的隱含知識,並且定義出不同隱含知識的探勘方式,Fayyad~\cite{fayyad1996data} 依照不同資料探勘技術之特性,分出迴歸、分類、分群、關聯四大類,本研究的第一階段即根據此四種知識形式,以及校舍耐震補強計畫的執行流程與需求,假設並定出各種可能透過資料探勘技術取得的校舍耐震資料庫隱含的知識以及探勘方法,其中,迴歸形式的可能取得知識包括了校舍耐震能力預測、校舍破壞模式預測、校舍補強經費預測等,分類形式的可能知識包括了校舍是否需要補強的預測,分群形式的知識則是校舍的類型歸類條件,關聯形式的可能知識則是校舍設計參數與其現狀的關連性。

第二階段則是根據假設的各種隱含知識和資料探勘規劃,實際進行資料探勘的分析和測試,最後的第三階段則是探勘結果的驗證和隱含知識的整理,基於此一流程,本研究得到了三個有一定可靠度的校舍耐震資料庫的隱含知識,分別為:

\begin{itemize}
\item 校舍資訊與耐震能力之關係模型
\item 校舍資訊與破壞構件之關係模型
\item 校舍資訊與補強經費之關係模型
\end{itemize}

校舍耐震能力關係模型為本研究最主要的資料探勘目標,因為校舍耐震能力補強計畫當中,物理上最重要的資訊就是校舍的耐震能力,傳統上,如果要取得可靠的校舍耐震能力,需要由專業的技師來評估,其過程需要先到現場調查,根據調查的資訊建立完整的結構數值模型,並使用非線性推垮分析,其過程耗時且所費不貲,因此現在校舍耐震能力補強計畫是以分階段篩選的機制,先讓所有校舍進行一個較為簡單的初步評估,再根據初步評估的結果來決定哪些校舍的耐震能力可能比較不足夠,需要詳細的非線性分析,判斷其是否安全,接著才真的對這些校舍進行詳細的非線性分析與耐震能力評估,然而這種方法有個缺點是其初步的評估方法無法完全反映出校舍的耐震能力,可能有些校舍已經年代久遠造成耐震能力低落,卻無法在初步評估的結果中真實的反映出來,因此,如果有一個方法可以快速的得到更為可靠的評估數據,甚至可以當作詳細評估的參考,將可以大大的加速校舍耐震能力補強計畫的進行。

除了數值化的校舍耐震能力,本研究還建立一個模型,可以找出校舍受到地震力時,優先破壞的構件進行,這個資訊可以幫助對校舍進行耐震能力評估的專業技師能對目標的校舍弱點先有一些初步了解,不但可以協助詳細評估的進行,對於校舍補強設計的方式也有一定程度的幫助。

最後,由於校舍補強所需的經費龐大,因此校舍耐震能力補強計畫不可能在短期內就把所有耐震能力不足的校舍都完成補強,實務上會需要估算各個校舍補強所需的經費,排定預算,然後才知道不同預算年度能夠完成多少的校舍補強作業,因此校舍的補強經費在教育部的計畫每年度初期的規劃時,是一個非常重要的數字,傳統的經費預估方法是由過往的經驗、數據和所欲補強校舍的規模作為依據,經由公式所計算出來的,如果能夠建立一個關係模型,經由校舍的基本資料得到更為準確的補強經費推估值,那便可以大大的加速校舍耐震能力補強計畫決策者的決策速度,提升決策品質,也可以讓計畫執行人員能更快的了解補強作業的規模。

\section{論文架構}

本論文共分為八章,各章內容分別介紹如下:

\begin{enumerate}
\item \textbf{\nameref{cha:intro}}:說明本研究之動機及目的。
\item \textbf{\nameref{cha:relative-works}}:回顧與本研究相關之文獻。
\item \textbf{\nameref{cha:database}}:介紹校舍耐震能力補強計畫之流程、校舍耐震資料庫之架構以及其所收集之資料內容。
\item \textbf{\nameref{cha:data-mining}}:介紹資料探勘技術以及本研究使用到的各種演算法以及評估方法,並依據資料探勘之知識類型分析探勘目標。
\item \textbf{\nameref{cha:ability}}:詳述本研究第一個使用資料探勘方法找到的隱含知識,校舍耐震能力與其設計參數間的關係模型。
\item \textbf{\nameref{cha:crack}}:詳述使用資料探勘技術找出校舍現況、設計參屬等屬性與其遇到地震力時,可能先受力破壞構件之關係模型之方法與過程。
\item \textbf{\nameref{cha:cob}}:詳述使用資料探勘技術找出校舍現況、設計參數等屬性與其所需之補強經費間關係模型之方法與過程。
\item \textbf{\nameref{cha:conclusions}}:結果探討與未來展望。
\end{enumerate}



