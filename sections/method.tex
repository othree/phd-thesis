\renewcommand\thetable{\arabic{chapter}-\arabic{table}}
%\renewcommand\thefigure{\arabic{chapter}-\arabic{figure}}
\renewcommand{\theequation}{\arabic{chapter}-\arabic{equation}}
\chapter{資料探勘}

資料探勘技術方法繁多,Fayyad~\cite{fayyad1996data}根據其處理的問題形式,將資料探勘的方法分為分類、分群、迴歸尋找關聯等四種主要的問題類型,分類方法處理的問題是在用來判斷資料的類別,而且這些類別是已知的類別,例如將所有的校舍資料分類成有安全疑慮和沒有安全疑慮的就是屬於分類問題。分群問題和分類問題有點相似,一樣是將資料分成數個群組,最主要的差異是分群問題的各個群組特性一開始並不清楚,分群方法是將資料根據其屬性數值為依據,把相似的放在同一個群組,不同群組的特性是要在分出群組後進行分析才會得到。迴歸問題就是要用回歸方法來從資料的屬性中,找出特定屬性與其他屬性間的關係模型,這些屬性間的關係可能是非線性的,而且沒有解析解的關係模型,因此常見的方法是用統計回歸的方式,用現有的資料來回歸得到,又或著是用像類神經網路之類的機器學習方式,拿現有的資料下去學習已得到關係模型,以校舍耐震資料庫來說,校舍耐震能力指標的預測就是一種回歸問題,因為校舍耐震能力指標與其校舍的設計參數間的關係就是一個非線性關係,要得到兩者之間的非線性模型就需要用到回歸問題的處理方法,回歸問題也是最常見的資料探勘問題種類。最後一種是尋找屬性間的關聯,這種問題的主要目標在尋找不同筆資料屬性間所存在的關係,舉例來說,使用校舍耐震資料庫的資料來作關聯分析,可能可以去尋找像是:五層樓的校舍的校舍長度深度有什麼趨勢,或是民國八十到九十年之間的校舍的校舍走廊設計是否偏好有走廊柱等。

本研究後期確定主要的探勘目標後,使用的資料探勘方式為迴歸為主,分類分群為輔助,以下分別介紹各種使用的的分析方法:

\section{Generalized Linear Model}

廣義線性模型是由Nelder and Wedderburn\cite{citeulike:5485398}所提出,比起迴歸分析(simple regression)更為彈性,此模型是假設資料點的分佈有一分佈模式,且X與Y之間的關係是由一連結函數(Link Function)建立,如log function、power function等,其定義之XY關係模型如下:

\begin{equation} g(E(y)) = x\beta + O, y~F \label{eq:GLM}\end{equation} 

$g(.)$是為所選的鏈結函數,O是偏移(offset)變數,F則是y的分佈模型,其是用牛頓法(Newton-Raphson Method)不斷的調整$\beta$使的$x\beta + O$逼近$g(E(y))$,最後最接近的方程式即為XY兩者的關系式。比起迴歸分析,此方法還需要了解Y值分佈狀況,選擇出最適合的分佈函數,並假設XY間的鏈結函數形式,雖然越多的參數選擇代表了更多的模型不確定性,但廣義線性模型卻能夠提供比迴歸分析更廣的應用範圍,也可能得到更接近真實的關係模型。

\section{Support Vector Machine}

SVM最早是BOSER\cite{boser1992}等人,在1992年的COLT (Computational Learning Theory)所提出,SVM是一個基於統計學習理論的分類方法,用來處理二元分割的問題,其原理是將原本無法線性分割的問題轉換到一個不同維度的空間(kernel)後,假設該空間存在一超平面(hyperplane),可以正確的將資料分開,並將尋找此一超平面的問題轉換為一最佳化問題,求解後即可得到二元分割邊界的方程式。而後Harris Drucker, et. al.,[9] 將此二元分割問題轉換為迴歸分析問題,故SVM也可以處理迴歸問題。

\section{Artificial Neural Networks}

其是希望能模擬建構出人腦內的神經網路,以處理各種複雜的問題,人類大腦是由大約千兆個神經元(Neuron)所構成,而每個神經元又會和其他約一萬個神經元連結,構成一個龐大且複雜的神經網路,這樣複雜的一個神經網路讓人類可以學習並了解各種事物與知識。McCulloch and Pitts ~\cite{mcculloch1943logical}所提出的模型為後續類神經網路發展的雛形,一個標準的類神經網路可以分為輸入層(input layer)、隱藏層(hidden layer)、輸出層(output layer),輸入層(input layer)負責接受各種求解問題需要的量化數據和資料,經由隱藏層(hidden layer)的不斷自我更新學習的模型處理過後,在輸出層(output layer)就可以得到想要的解答,類神經網路可以處理的問題種類多樣,其模型的品質多數也都不錯,缺點是學習時間長,且得到的模型為一個黑盒子,難以解釋其物理或是數學模型上的意義。

\section{Genetic Programming}



\section{Weighted Genetic Programming}



\section{K-means}

As proposed by MacQueen~\cite{macqueen67}, K-means is one of the most common clustering methods and has a wide application scope. Notably, it is a machine learning method; its principal steps are as follows.

\begin{enumerate}
\item A user indicates that data should be grouped into K clusters.
\item Data are divided randomly and equally into K groups and the center of each cluster are calculated.
\item Each bit of data should find the proximal center of a cluster and update its cluster label that it belongs to.
\item Recalculate the new cluster center.
\item Repeat steps 3 to 4 until the cluster centers of all data do not change.
\end{enumerate}

\section{Two-Step Classification}

Based on of the massive volume of basic data for school build- ings in the database, this study chooses two-step clustering method. The basic concept was first proposed by Zhang, Ramakrishnan and Livny~\cite{zhang1996birch} for handling large amounts of data. This method has two major steps. The first step sequences data and pre-clusters sequences into small subclusters based on the similarity of adjacent data, thereby reducing the amount of data. The second step divides several small subclusters into the desired number of clusters using a hierarchical clustering method. The hierarchical clustering method then combines close subclusters slowly until the stop condition is met. The computing speed of this method is influenced slightly by the volume of data.

