\renewcommand\thetable{\arabic{chapter}-\arabic{table}}
%\renewcommand\thefigure{\arabic{chapter}-\arabic{figure}} 
\chapter{校舍資訊與破壞構件之關係模型}

校舍耐震評估最準確的階段為進行詳細評估作業,但進行校舍詳細評估須歷經非常多之流程如圖 1.2,花費很多時間,如能有一輔助工具能於執行校舍詳細評估前期即能粗估校舍之可能破壞構件與其破壞模式,將有助於土木技師與專業人員進行作業時能先有初步結果之掌握,與校方或主管機關進行洽談時也能更加迅速。依前段之敘述,本研究之子主題(預測破壞構件及其模式)主要是希望能於校舍詳細評估初期能有一些關於詳細評估結果之初步掌握,因此研究將採取耐震評估作業中花最少經費的階段(校舍普查作業)與詳細評估之評估結果,並以鋼筋混凝土建築、單邊走廊且廊外無柱、無地下室、詳細評估顯示為已送出之資料做為此模型資料選擇之條件篩選,藉由已存在校舍資料庫內之大量資料,以找出期間關聯性達成研究之預測目的。

\section{資料前處理}

本預測模型其主要功能為輔助土木技師與專業人員於執行校舍詳細評估前期即能粗估校舍之可能破壞構件與其破壞模式。校舍耐震評估最準確的階段為進行詳細評估作業,但進行校舍詳細評估須歷經非常多之流程如圖 1.2,花費很多時間,藉由此模型之預測輔助,將有助於土木技師與專業人員進行作業時能先有初步結果之掌握,與校方或主管機關進行洽談時也能更加迅速。

依本章~3-2~節敘述,本研究需使用校舍資料庫內校舍普查與詳細評估這兩個資料表,經連結兩表後發現,資料庫內並無記錄校舍跨距之資料,因此本研究依據此規則「校舍長度/(一樓教室柱/2-1)」計算創造出一新欄位,另外校舍之長度與深度亦有填反之現象,因此本研究透過人工方式將此現象做修正。

本模型採取耐震評估作業中花最少經費的階段(校舍普查作業)與詳細 評估之評估結果,並以鋼筋混凝土建築、單邊走廊且廊外無柱、無地下室、詳細評估顯示為已送出之資料做為此模型資料選擇之條件篩選,資料經條件篩選完後,開始進行探勘模型之屬性欄位選擇,本模型挑選之 屬性集如圖 4.3,選擇之欄位分為輸入變數與輸出變數,本模型之輸出變數為破壞構件與其破壞模式,輸入變數則是我們用來預測輸出變數之所用變數,表 4.8 為本預測模型屬性集內之屬性介紹與選擇說明。

\begin{multicols}{2}
\begin{itemize}
\item 校舍長度
\item 校舍深度
\item 樓層數
\item 地上層樓地板總面積
\item 一樓教室柱根數
\item 一樓走廊外柱斷面積和
\item 475~年設計地表加速度
\item 校舍垮距
\item 用途係數
\item X~向~1F~三面圍束磚牆斷面積
\item 有無三面圍束磚牆
\item 有無四面圍束磚牆
\item 有無磚牆
\item 同時有三面和四面圍束磚牆
\item 有無~RC~牆
\item[]
\end{itemize}
\end{multicols}

\begin{description}
  \item[校舍長度]
  中小學校舍中之典型校舍其建築形式較為固定且普遍存有耐震能力較弱之現象,因此本研究推測在這些典型校舍中,某些特定尺寸(校舍長度)之校舍應與其極限強度下會破壞之構件與其破壞模式存有一定之相關性。因此本研究將此屬性納入屬性集做為本探勘模型之輸入屬性。
  \item[校舍深度]
  中小學校舍中之典型校舍其建築形式較為固定且普遍存有耐震能力較弱之現象,因此本研究推測在這些典型校舍中,某些特定尺寸(校舍深度)之校舍應與其極限強度下會破壞之構件與其破壞模式存有一定之相關性。因此本研究將此屬性納入屬性集做為本探勘模型之輸入屬性。
  \item[樓層數]
  加入此屬性,其構想為假設這些校舍中應該有集中幾層樓之建築與其極限強度下會破壞之構件與其破壞模式存有一定之相關性。因此本研究將此屬性納入屬性集做為本探勘模型之輸入屬性。希望藉由其中隱藏之關聯性達到目標之預測。
  \item[地上層樓地板總面積]
  校舍之樓地板面積亦屬於建築形式之一環,因此可視為該校舍之特有形式,本研究假設在典型校舍中存有特定樓地板面積大小有其普遍對應之耐震能力或某些尺寸之樓地板面積可能存有接受補強較高的機率。
  \item[一樓教室柱根數]
  教室柱之根數間接可之校舍樓地板面積之大小或規模,因此本研究將之視為影響工程經費之因素之一,經將此屬性納入探勘模型中測試後,也確實發現此屬性對預測之正確率有一定之提升。
  \item[一樓走廊外柱斷面積和]
  台灣中小學校舍其建築形式較為固定,教室柱之多寡或其尺寸亦可視為建築形式中之一環,因此本研究推測在這些校舍中,教室柱應該存有某些特定總斷面積大小之校舍其耐震能力較差,因此本研究將此屬性納入屬性集做為本探勘模型之輸入屬性。
  \item[475~年設計地表加速度]
  由校舍所在位於查出其工址短週期設計水平譜加速度之 0.4 倍($0.4S_{DS}$)
  \item[校舍垮距]
  資料庫內並無記錄校舍跨距之資料,此屬性為本研究依據「校舍長度/(一樓教室柱/2-1)」計算創造出之一新欄位。
  \item[用途係數]
  該屬性為紀錄該校舍是否作為緊急避難使用,以決定其值為 1.25 或 1.5,為反映其因重要性所需之安全係數。如該校舍當初規劃為緊急避難使用,那其當初設計強度就會比較高,本研究設想不同之用途係數其可能受到之破壞模式可能有某種關係,因此將之加入本屬性集內。
  \item[X~向一樓三面圍束磚牆總斷面積]
  磚牆對校舍能提供部分耐震能力,本研究試過單獨放入三面圍束、四面圍束與同時放入三面與四面圍束磚牆,經分別試驗不同之屬性集,建立多個不同的模型,其結果以單獨使用三面圍束磚牆總斷面積做為輸入屬性,其預測之結果最佳。
  \item[有無三面圍束磚牆]
  此屬性為校舍資料庫不存在之欄位,因經測試許多屬性集與不同探勘方法發現磚牆對預測結果有一定影響力,但並不是所有校舍均有磚牆,因此本研究規劃創造出一新欄位,紀錄有無三面圍束磚牆,其內之值為依照原三面圍束磚牆欄位內有值且非 0 或 NULL,其產生之新欄位,其值設為 1,相反則設為 0。
  \item[有無四面圍束磚牆]
  此屬性為校舍資料庫不存在之欄位,因經測試許多屬性集與不同探勘方法發現磚牆對預測結果有一定影響力,但並不是所有校舍均有磚牆,尤其是四面圍束磚牆,因此本研究規劃創造出一紀錄有無四面圍束磚牆之新欄位,期望能對此模型之預測目標提升一定之正確率,該屬性之值為依照 原四面圍束磚牆欄位內有值且非 0 或NULL,其產生之新欄位,其值設為 1,相反則設為 0。
  \item[有無磚牆]
  此屬性為校舍資料庫不存在之欄位,因經測試許多屬性集與不同探勘方法發現磚牆對預測結果有一定影響力,但並不是所有校舍均有磚牆,且磚牆有分三面圍束與四面圍束,經測試單獨存有三面圍束與四面圍束其預測結果均不同,因此本研究規劃創造出一新欄位,紀錄校舍有無磚牆,其內之值為依照原三面圍束與四面圍束磚牆欄位,只要任一欄位內有值且非 0 或 NULL,其產生之新欄位,其值設為 1,相反則設為 0。
  \item[同時有無三面和四面圍束磚牆]
  此屬性為校舍資料庫不存在之欄位,因經測試許多屬性集與不同探勘方法發現磚牆對預測結果有一定影響力,經測試校舍是否同時存有三面圍束與四面圍束磚牆對其預測結果,存有一定影響力。因此本研究規劃創造出一新欄位,紀錄校舍有無同時建有三面和四面圍束磚牆,其內之值為依照原三面圍束磚牆與四面圍束磚牆欄位內,同時有值且非 0 或 NULL,其產生之新欄位,其值 設為 1,相反則設為 0。
  \item[有無~RC~牆]
  此屬性為校舍資料庫不存在之欄位,因經測試許多屬性集與不同探勘方法發現 RC 牆對預測結果有一定影響力,但並不是所有校舍均有 RC 牆,因此本研究規劃創造出一新欄位,紀錄有 RC 牆,其內之值為依照原 RC 牆欄位內有值且非 0 或 NULL,其產生之新欄位,其值設為 1,相反 則設為 0。
\end{description}

\section{資料探勘}

本研究採用支援向量機進行資料探勘之建模,選擇此探勘方法,主因為本研究之預測目標,破壞構件及其破壞模式其資料類型為「名義」, 名義之資料類型其最適合之探勘方法為支援向量機,因此本研究採用此探勘方法,使用前處理完之 897 筆資料並挑選屬性集內之欄位,將整體資料分為訓練區 70\%與測試區 30\%,做為探勘模型之訓練方法。最後再挑一組結果較好之模型保留,並以另外挑選之獨立 50 筆校舍資料進行結果測試,測試結果敘述於第五章。本模型之訓練結果如表~\ref{tab:comp_result}。

\setlength{\tabcolsep}{2em}
\begin{table}[hbtp]
  \begin{center}
    \caption{Result of the Crack Component Model}
    \label{tab:comp_result}
    \begin{tabular}{l c c}
    	\hline
    	破壞元件 & 資料集 & 正確率 \\
    	\hline
    	\multirow{2}{*}{X 正向樑破壞} & 訓練集 & 76.97\% \\
    	\cline{2-3} & 測試集 & 82.14\% \\
    	\hline
    	\multirow{2}{*}{X 正向柱破壞} & 訓練集 & 86.52\% \\
    	\cline{2-3} & 測試集 & 86.90\% \\
    	\hline
    	\multirow{2}{*}{X 正向窗台柱破壞} & 訓練集 & 79.21\% \\
    	\cline{2-3} & 測試集 & 77.38\% \\
    	\hline
    	\multirow{2}{*}{X 正向~RC~牆破壞} & 訓練集 & 99.32\% \\
    	\cline{2-3} & 測試集 & 98.25\% \\
    	\hline
    	\multirow{2}{*}{X 正向磚牆破壞} & 訓練集 & 80.34\% \\
    	\cline{2-3} & 測試集 & 80.95\% \\
    	\hline
    \end{tabular}
  \end{center}
\end{table}

\section{結果}

本探勘模型使用支援向量機並分別針對耐震詳細評估資料表之性能點狀態下最嚴重破壞樓層之主要破壞桿件及其破壞模式進行預測,其各項之預測正確率如下: \\ \indent
X~正向樑破壞 97.06\% \\ \indent
X~正向柱破壞 100\% \\ \indent
X~正向窗台柱破壞 94.12\% \\ \indent
X~正向 RC 牆破壞 100\% \\ \indent
X~正向磚牆破壞 82.6\% \\ \indent
最後並根據每棟校舍於性能點狀態下,各項破壞構件其破壞模式進 行正確率之計算,根據耐震詳細評估資料表內有五項破壞構件之欄位,因此本研究假設該校舍於每項破壞構件之欄位如預測正確,則加正確率百分之二十,共五項欄位,如各項接預測正確,則該校舍針對其性能點狀態下最嚴重破壞樓層之主要破壞桿件及其破壞模式之預測正確率為百分之百。其 34 棟總體校舍針對 5 種破壞構件及其破壞模式之平均預測正確率為 95.3\%。本預測目標其欄位屬性為類別型態,因此無法以其他評估指標進行比較,一般評估指標只能用來評估數值型資料。34 棟校舍於破壞構件及其破壞模式之預測資料與結果如,X 正向樑破壞於表 5.6 中欄位名稱為 D\_pXdestoryB,X 正向柱破壞於表 5.6 中欄位名稱為~D\_pXdestoryC,X 正向窗台柱破壞於表 5.6 中欄位名稱為 D\_pXwindowsC, X 正向 RC 牆破壞於表 5.6 中欄位名稱為 D\_pXRCwall,X 正向磚牆破壞於表 5.6 中欄位名稱為 D\_pXBwall,其餘之欄位說明如表 5.5。
